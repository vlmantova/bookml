%%% common LaTeX classes will work fine
\documentclass[oneside,11pt]{amsart}

%%% recommended for PDF quality: better handling of accented characters and ligatures
\usepackage[utf8]{inputenc}
\usepackage[T1]{fontenc}

%%% REQUIRED for accessibility: set the language for the output (here 'en-GB')
\usepackage[british]{babel}

%%% REQUIRED for correct formatting and tagging of BookML output: title and author
\title{Title}
\author{Author}
\date{Date}

%%% RECOMMENDED for PDF quality: enable clickable links and metadata in PDFs
% the option pdfusetitle adds \author and \title to the PDF metadata
\usepackage[pdfusetitle]{hyperref}

%%% RECOMMENDED for PDF quality: higher quality equivalent of Computer Modern
\usepackage{lmodern}

%%% OPTIONAL: additional BookML and LaTeXML functionality
\usepackage{bookml/bookml}

%%% REQUIRED for TikZ: use bmlImageEnvironment for TikZ and other images, since
%%%                    LaTeXML is slow and often produces garbled TikZ output
%%%                    (this requires \usepackage{bookml/bookml})
% tell BookML to compile tikzpicture and tikzcd environments into images via LaTeX
\bmlImageEnvironment{tikzpicture,tikzcd}
% hide all TikZ-related commands from LaTeXML, but leave them when generating the PDF
\iflatexml            % are we compiling through LaTeXML?
\else
\usepackage{tikz}     % if not, load TikZ as usual
\usetikzlibrary{cd}
\usepackage[all]{xy}
% other TikZ-related command MUST be here, before \fi
\fi

%%% OPTIONAL for accessibility: additional versions of the document (e.g. large print)
% the additional formats will be recompiled automatically whenever there are changes
% you may use "\jobname" instead of "template" to pick up the name from this very file
\bmlAltFormat{template.pdf}{PDF (serif)} % always included; here \bmlAltFormat simply changes its label
\bmlAltFormat{template-sans.pdf}{PDF (sans serif)}
\bmlAltFormat{template-sans-large.pdf}{PDF (sans, large)}

% very cheap way of generating a 'large print' PDF (~19% bigger)
\ifcsname bmlCrop\endcsname % is \bmlCrop defined?
\usepackage{crop}           % if so, load the package crop
\fi
%%% What is \bmlCrop?
% template-sans-large.tex uses \documentclass[oneside,11pt]{amsart}

%%% RECOMMENDED
% better accented letters and ligatures in PDF
\usepackage[T1]{fontenc}
% set language of outputs
\usepackage[british]{babel}
% PDF hyperlinks and metadata
\usepackage[pdfusetitle]{hyperref}
% higher quality variant of Computer Modern
\usepackage{lmodern}
% title and author
\title{BookML template}
\author{Author}

%%% OPTIONAL
% additional BookML and LaTeXML functionality
\usepackage{bookml/bookml}
% additional versions of the document (e.g. large print) for inclusion in the HTML output
\bmlAltFormat{\jobname.pdf}{PDF (serif)} % this version is always included unless it's given an empty name
\bmlAltFormat{\jobname-sans.pdf}{PDF (sans serif)}
\bmlAltFormat{\jobname-sans-large.pdf}{PDF (sans, large)}
% simple way of generating a large print PDF (~19% bigger) without changing pagination
\ifcsname bmlCrop\endcsname\usepackage{crop}\fi
% short text description used for PDF and SCORM metadata
\hypersetup{pdfsubject={Empty template for use with BookML.}}

\begin{document}

\maketitle

You can use this template in the following ways:
\begin{enumerate}
  \item Replace the content of \texttt{template.tex} with your content. To change the file names in the output, rename:
  \begin{itemize}
      \item \texttt{template.tex} to \texttt{NAME.tex},
      \item \texttt{template-sans.tex} to \texttt{NAME-sans.tex} (and adjust its content),
      \item \texttt{template-sans-large.tex} to \texttt{NAME-sans-large.tex} (and adjust its content).
  \end{itemize}
  \item Add \LaTeX{} files to the same folder containing \texttt{template.tex}. Any files containing the string \verb|\documentclass| (even commented out) will be compiled separately. You may delete \texttt{template.tex} and accompanying \texttt{sans} and \texttt{sans-large} files if you are not using them.
\end{enumerate}
In all cases, consult \texttt{template.tex} for how to use more advanced BookML functionality (such as providing large print PDFs or setting additional PDF and SCORM metadata).

\bigskip

For use with Overleaf (requires paid account):
\begin{enumerate}
  \item Upload the template to a new Overleaf project. Ensure that the \texttt{.github} folder is also included.
  \item From the Overleaf menu, sync the project with a new (private) GitHub repository. After one or two minutes, you will receive an email announcing that the repository has a new a release containing the various BookML outputs. Follow the link in the email to download them.
  \item Every time you sync your Overleaf changes to GitHub, a new build will start and generate a new release.
  \item For a moderate speed up, change the \TeX{} Live scheme in the file \texttt{.github/workflows/bookml.yaml}. This will download smaller \TeX{} Live images containing fewer packages.
  \item If the build is cancelled, increase the timeout in the same file. See the \href{https://github.com/vlmantova/bookml-action}{BookML action documentation} for more details.
\end{enumerate}

%%% OPTIONAL
% footer for HTML output (will not appear in the PDF)
\begin{lxFooter}
  Copyright \copyright{} 2025 Author. Compiled with BookML \BookMLversion.
\end{lxFooter}

% copyright notice for PDF output
\iflatexml\else
\bigskip
\begin{center}
  {\small Copyright \copyright{} 2025 Author. Compiled with BookML \BookMLversion.}
\end{center}
\fi

\end{document}
 to read this file.
% Before that, it defines \bmlCrop, so that the above code imports crop.
% It also uses \PassOptionsToPackage to tweak the options of crop.
% The result has identical pagination as the normal PDF so you do not need
% to review both versions.
%
% However, the magnification is very limited. If you need larger magnification,
% you may have to use a larger font, in which case the large print PDF will
% will look different and must be reviewed regularly.

% rest of preamble

\begin{document}

%%% RECOMMENDED for SCORM: the abstract becomes the description of the SCORM package
\begin{abstract}
  Abstract (will appear as description on your LMS).
\end{abstract}

\maketitle

%%% OPTIONAL: footer for HTML output, for instance a copyright notice
% note that the footer does not appear in the PDF
% the PDF footer can be done in any of the usual ways (e.g. fancyhdr)
\begin{lxFooter}
  Footer: copyright \copyright{} 2022-23 Author.
\end{lxFooter}

Content (you may also \texttt{\textbackslash{}input} or
\texttt{\textbackslash{}include} other files, including bibliographies).

A formula:
\[ e^{\pi i} = -1. \]

\end{document}
