\RequirePackage{bookml/bookml}
\documentclass[a4paper,british]{article}

\usepackage[T1]{fontenc}
\usepackage[utf8]{inputenc}

\usepackage{latexml}

\usepackage{babel}

\usepackage[pdfusetitle,colorlinks]{hyperref}
\usepackage{ifpdf}
\ifpdf % make build reproducible
\pdfinfoomitdate=1
\pdftrailerid{}
\pdfsuppressptexinfo=-1
\hypersetup{pdfinfo={ Creator={}, Producer={} }}
\fi

\usepackage[dvipsnames]{xcolor}
\usepackage{geometry}
\setlength{\parindent}{0pt}
\setlength{\parskip}{0.3em}

\iflatexml
\def\Xy{Xy}
\else
\usepackage[all]{xy}
\usepackage{tikz}
\usetikzlibrary{ducks}
\usepackage{animate}
\fi
\def\tikzname{Ti\emph{k}Z}

\bmlImageEnvironment{animateinline}

\usepackage{listings}
\colorlet{WCAGGreen}{OliveGreen!80!black} % a bit darker to get sufficient contrast
\lstset{basicstyle={\small\ttfamily},%
  keywordstyle={\color{blue}\bfseries},%
  keywordstyle=[2]{\color{MidnightBlue}\bfseries},%
  stringstyle={\color{red}},%
  commentstyle={\color{WCAGGreen}},%
  frame=none,
  xleftmargin=1em,xrightmargin=1em%
}

\lstdefinestyle{bookml}{language=[LaTeX]TeX,%
  texcsstyle=*{\color{blue}\bfseries},%
  texcsstyle=*[2]{\color{MidnightBlue}\bfseries},%
  moretexcs={xymatrix,ltxinline,href,RequirePackage},%
  moretexcs=[2]{LaTeXML,iflatexml,lxAddClass,lxWithClass,lxBeginTableHead,lxEndTableHead,lxFcn,lxID,lxPunct,lxContextTOC,lxNavbar,lxHeader,lxFooter,bmlImageEnvironment,bmlHTMLEnvironment,bmlRawHTML,bmlDescription,bmlPlusClass,bmlHTMLInlineEnvironment}%
  }

\def\ltxinline{\lstinline[style=bookml]}
\def\htmlinline{\lstinline[language=html]}

\usepackage{amsmath}

\title{BookML: a bookdown flavoured GitBook port for \texorpdfstring{\LaTeXML{}}{LaTeXML}}
\author{Vincenzo Mantova}

\date{21 November 2021}

\begin{document}

\maketitle

\begin{abstract}
  BookML is a small add-on for \LaTeXML{} with a few accessibility features and quality of life improvements. Its purpose is to simplify the conversion of \LaTeX{} documents into \HTML{} files that conform to the \href{https://www.w3.org/TR/WCAG21/}{Web Content Accessibility Guidelines 2.1} level AA.

  The key features:
  \begin{itemize}
    \item a port of the \href{https://bookdown.org/yihui/bookdown/html.html#gitbook-style}{GitBook style} of \href{https://bookdown.org}{bookdown}, tweaked for better WCAG conformance; this is enabled by default, but can be disabled;
    \item styling fixes for \LaTeXML{} (many backported from 0.8.6), such as mobile friendly responsive output;
    \item transparent generation of SVG pictures via \LaTeX{} for packages not well supported by \LaTeXML{}, such as \tikzname{} pictures, animations, \Xy-matrices;
    \item a simple method to add alternative text for images;
    \item partial support for arbitrary \HTML{} content;
    \item direct embedding of MathJax, with the option of choosing between versions 2 and 3 or disabling it.
  \end{itemize}
\end{abstract}

\begin{center}
  Formats: \href{https://vlmantova.github.io/bookml/}{GitBook} (html), \href{https://vlmantova.github.io/bookml/index.plain.html}{plain} with Computer Modern (html), \href{https://vlmantova.github.io/bookml/docs.pdf}{PDF}, \href{https://vlmantova.github.io/bookml/docs.epub}{EPUB}, \href{https://github.com/vlmantova/bookml/blob/docs/docs.tex}{\LaTeX{} source}.
\end{center}

\tableofcontents

\section{Getting started}

\begin{enumerate}
  \item Install the prerequisites:
  \begin{itemize}
    \item \textbf{Required:} a working \href{https://dlmf.nist.gov/LaTeXML/}{LaTeXML} installation, at least version 0.8.5.
    \item \textbf{Optional} (necessary for generating the images via \LaTeX): a working \TeX{} install containing \lstinline|dvisvgm|, \lstinline|latexmk|, \lstinline|preview.sty|.
  \end{itemize}
  \item {}[Install/upgrade] Unpack the latest \href{https://github.com/vlmantova/bookml/releases}{BookML release} and put the \lstinline|bookml| folder next to your \lstinline|.tex| files.
  \item {}[First install only] Copy the files \lstinline|bookml/XSLT/LaTeXML-*.xsl| next to your \lstinline|.tex| files.
  \item \label{edit-tex-file} Anywhere between \ltxinline|\documentclass| and \ltxinline|\begin{document}|, add \ltxinline|\usepackage{bookml/bookml}|. Alternatively, you can also add \ltxinline|\RequirePackage{bookml/bookml}| before \ltxinline|\documentclass| (for instance, if you want to use a different class when compiling with \LaTeXML{}).
  \item Compile your file to HTML:
    \begin{lstlisting}
latexmlc --navigationtoc=context \
  --dest=my_latex_file/index.html \
  my_latex_file.tex
    \end{lstlisting}
    Add \ltxinline|--splitat=section| (or chapter, part...) to split the content on multiple pages. See the \LaTeXML{} \href{https://dlmf.nist.gov/LaTeXML/manual/usage/}{manual} for all the options.
  \item Compile your file to EPUB3 (experimental, requires patched \LaTeXML{} 0.8.6):
    \begin{lstlisting}
latexmlc --preload='[style=plain]bookml/bookml' \
  --dest=my_latex_file.epub \
  --splitat=section \
  my_latex_file.tex
    \end{lstlisting}
\end{enumerate}

If you do not want to modify your \lstinline|.tex| in step~\ref{edit-tex-file}, add the option \lstinline|--preload=bookml/bookml| when calling \lstinline|latexml| or \lstinline|latexmlc|.

You can also use bookml directly from source:
\begin{lstlisting}
git clone https://github.com/vlmantova/bookml.git
cd bookml && make
\end{lstlisting}
then copy the files \lstinline|bookml/XSLT/LaTeXML-*.xsl| next to your \lstinline|.tex| files.

\section{Options}

The \ltxinline|bookml| package accepts a few options (for instance use \ltxinline|\usepackage[style=plain,nomathjax]{bookml/bookml}| to disable the GitBook style and to avoid including MathJax). The options have no effect on the PDF output.

You can also pass these options at compilation time using the \ltxinline|--preload| option of \ltxinline|latexml| and \ltxinline|latexmlc|:
\begin{lstlisting}
  latexmlc --preload=[style=plain,nomathjax]bookml/bookml \
    --dest=my_latex_file/index.html \
    my_latex_file.xml
\end{lstlisting}

\begin{description}
  \item[\texttt{style=gitbook}] Use the GitBook style (the default behaviour). When using the GitBook style, you must call \ltxinline|latexmlc| (or \ltxinline|latexmlpost|) with the option \ltxinline|--navigationtoc=context|. Any PDF or EPUB file with the same name as the source will be detected and added to the download menu.
  \item[\texttt{style=plain}] Use the \LaTeXML{} style with a few slightly opinionated tweaks.
  \item[\texttt{style=none}] Use the \LaTeXML{} style with no tweaks (except for backported styles and some fixes).
  \item[\texttt{nomathjax}] Do not include MathJax in the output.
  \item[\texttt{mathjax=2}] Use MathJax version 2 instead of version 3.
  \item[\texttt{imagescale=X.XXX}] Rescale the images generated via \LaTeX{} (\S~\ref{sec:external-image}) by the desired factor. The scaling factor is adjusted internally based on the options \ltxinline|8pt|, \ltxinline|9pt|, \dots, \ltxinline|12pt| being passed to the document class.
\end{description}

\section{Customisation}

\subsection{\CSS{} and fonts}
Just create a \lstinline|bmluser| folder and add any \lstinline|.css| file to it. The files will be included at the end of the \htmlinline|<head>| tag and override the previous styles.

If the file name ends with \lstinline|.style1,style2.css|, then that file will be used only when \lstinline|style=style1| or \lstinline|style=2| is passed. You can use \lstinline|._all.css| to ensure that the file is included in every style.

The \href{https://vlmantova.github.io/bookml/index.plain.html}{plain} version of this document has been compiled with an additional \href{https://vlmantova.github.io/bookml/bmluser/computer-modern.plain.css}{\texttt{computer-modern.plain.css}} that sets the font to Computer Modern.

\subsection{\HTML{} output}

You can modify \lstinline|LaTeXML-html5.xsl| to change the \HTML{} output. Just be careful not to overwrite the file during updates.

\section{Commands}

\subsection{Conditional execution}
Call \ltxinline|\iflatexml ... \else ... \fi| to write code that is executed only by \LaTeXML{}, or only (pdf)\LaTeX{} respectively. \lstinline|bookml| will try to use the \lstinline|latexml| package, if available, or use its own embedded copy. See Figure~\ref{fig:iflatexml}.

\begin{figure}[hb]
  \begin{lstlisting}[style=bookml]
\caption{Example of
  \iflatexml\ltxinline|\xymatrix|\else\ltxinline|\\xymatrix|\fi{}
  from the \ltxinline|xypic| documentation.}
  \end{lstlisting}
  \caption{Example of \texttt{\textbackslash{}iflatexml}, used in Figure~\ref{fig:xymatrix} to work around a subtle difference between \LaTeXML{} and \LaTeX{}.}
  \label{fig:iflatexml}
\end{figure}

\subsection{Alternative text for images}
Call \ltxinline|\bmlDescription{textual description}| \emph{right after} an image to populate its \htmlinline|alt| attribute (or \ltxinline|aria-label| if appropriate). Inspect the \HTML{} source of Figure~\ref{fig:duck} or use a screen reader to check its text description.

\subsection{Disable MathJax for some equations}
Call \ltxinline|\bmlDisableMathJax{}| inside an equation to stop MathJax from processing the equation. Useful if you want to use MathJax, but you have equations that are better handled by the browser. When used inside an environment creating multiple equations, it applies only to the ones containing the command. See Figure~\ref{fig:disable-mathjax}.

\begin{figure}
  \begin{lstlisting}[style=bookml]
\begin{align*}
  \int_0^{+\infty} x^2 \mathrm{d}x \quad \text{rendered by \href{https://www.mathjax.org/}{MathJax}} \\
  \bmlDisableMathJax \int_0^{+\infty} x^2 \mathrm{d}x \quad \text{rendered by \emph{the browser}}
\end{align*}
  \end{lstlisting}
  \begin{align*}
    \int_0^{+\infty} x^2 \mathrm{d}x \quad \text{rendered by \href{https://www.mathjax.org/}{MathJax}} \\
    \bmlDisableMathJax \int_0^{+\infty} x^2 \mathrm{d}x \quad \text{rendered by \emph{the browser}}
  \end{align*}
  \caption{How to disable MathJax for a single equation.}
  \label{fig:disable-mathjax}
\end{figure}

\subsection{Add custom CSS classes}
Call \ltxinline|\bmlPlusClass{class}| \emph{right after} some piece of content to add a CSS class. If done within text, its effect may be unpredictable. Its main use is to call \ltxinline|\bmlPlusClass{bml_no_invert}| after an image to prevent the picture from getting inverted in night mode. Compare how Figure~\ref{fig:duck} (with \ltxinline|bml_no_invert|) and Figure~\ref{fig:xymatrix} (no additional classes) change in night mode to see the difference.

Note that the package \ltxinline|latexml| also offers \ltxinline|\lxAddClass| and \ltxinline|\lxWithClass| for the same effect but different behaviour regarding which element gets the class.

\subsection{Generate pictures with \LaTeX{}}
\label{sec:external-image}

LaTeXML supports the \ltxinline|picture| environment as well as \emph{some} \tikzname{} pictures, but not all which will come out mangled, and some common packages are not supported altogether (for instance \ltxinline|xypic|, \ltxinline|tikzcd|, \ltxinline|animate|).

BookML offers a simple automated way of generating SVG images using \LaTeX{}, bypassing LaTeXML entirely. In your preamble, after \ltxinline|\usepackage{bookml/bookml}|, write
\begin{lstlisting}[style=bookml]
\bmlImageEnvironment{tikzpicture}
\bmlImageEnvironment{animateinline}

% optional, but strongly recommended:
% do not load tikz when running in LaTeXML
\iflatexml\else
\usepackage{tikz}
\usepackage{animate}
\fi
\end{lstlisting}

Now every \ltxinline|tikzpicture| and \ltxinline|animate| environment will be compiled with \LaTeX{} (using \lstinline[frame=none]|latexmk|) and converted to SVG images (using \lstinline[frame=none]|dvisvgm|). Figure~\ref{fig:duck} demonstrates this approach.

\begin{figure}
  \begin{lstlisting}[style=bookml]
\begin{animateinline}[
  alttext=none,loop,controls,nomouse,poster=20,autopause,
  begin={\begin{tikzpicture}
    \useasboundingbox (0,0) rectangle (4,3);},
    end={\end{tikzpicture}}]{30}
  \multiframe{160}{dShift=40mm+-0.5mm}
    {\duck[tophat,xshift=\dShift]}
\end{animateinline}
\bmlDescription{A stylised rubber duck, yellow
  and wearing a black top hat, enters from the right
  and slides until it exits from the left. The animation
  repeats every six seconds.}
\bmlPlusClass{bml_no_invert} % preserve colours in night mode
  \end{lstlisting}
  \begin{center}
    \begin{animateinline}[
      alttext=none,loop,controls,nomouse,poster=20,autopause,
      poster=20,autopause,
      begin={\begin{tikzpicture}%
        \useasboundingbox (0,0) rectangle (4,3);},
        end={\end{tikzpicture}}]{30}
      \multiframe{160}{dShift=40mm+-0.5mm}
        {\duck[tophat,xshift=\dShift]}
    \end{animateinline}
  \end{center}
  \bmlDescription{A stylised rubber duck, yellow
    and wearing a black top hat, enters from the right
    and slides until it exits from the left. The animation
    repeats this every six seconds.}
  \bmlPlusClass{bml_no_invert} % preserve colours in night mode
  \caption{A fancy duck. Click on the play button to start the animation (for the PDF, it requires a compatible software such as Acrobat Reader).}
  \label{fig:duck}
\end{figure}

If you only need this mechanism in a pinch, you can simply wrap the desired content between \ltxinline|\begin{bmlimage}| and \ltxinline|\end{bmlimage}| as exemplified in Figure~\ref{fig:xymatrix}.
\begin{figure}
  \begin{lstlisting}[style=bookml]
\begin{bmlimage}
  \[ \xymatrix{
      U \ar@/_/[ddr]_y \ar@/^/[drr]^x \ar@{.>}[dr]|-{(x,y)} \\
      & X \times_Z Y \ar[d]^q \ar[r]_p & X \ar[d]_f \\
      & Y \ar[r]^g & Z} \]
\end{bmlimage}
  \end{lstlisting}
  \begin{bmlimage}
    \[ \xymatrix{
        U \ar@/_/[ddr]_y \ar@/^/[drr]^x \ar@{.>}[dr]|-{(x,y)} \\
        & X \times_Z Y \ar[d]^q \ar[r]_p & X \ar[d]_f \\
        & Y \ar[r]^g & Z} \]
  \end{bmlimage}
  \bmlDescription{Universal property of the pullback as defined in [omitted].}
  \caption{Example of \iflatexml\ltxinline|\xymatrix|\else\ltxinline|\\xymatrix|\fi{} from the \ltxinline|xypic| documentation.}
  \label{fig:xymatrix}
\end{figure}

\subsection{Direct \HTML{} input}

You can insert arbitrary \HTML{} code using \ltxinline|\bmlRawHTML{html code}|.

\textbf{Warning:} the \HTML{} code needs to be written in `XML syntax', so you have to close all the tags (for instance, write \htmlinline|<br/>| instead of \htmlinline|<br>|, close the \htmlinline|<p>| tags, and so on) and empty attributes \emph{must} be given the value \htmlinline|""| (see this \href{https://dev.w3.org/html5/html-author/#elements}{old W3C guide} for some indications). Moreover, you must remember to escape your \ltxinline|%&_^${}|, and replace \ltxinline|\| with \ltxinline|\textbackslash|.

\ltxinline|\bmlRawHTML| is robust, i.e.\ it does not change the category codes, so it can be used inside \ltxinline|\newcommand| to create custom macros. See for instance Figure~\ref{fig:youtube-ducks} for a generic YouTube embedding macro. Note that the video will not be visible in the PDF, so a link should always be provided (possibly PDF only, as in the example).

\begin{figure}
  \begin{lstlisting}[style=bookml]
\newcommand{\youtube}[2]{\bmlRawHTML{
  <div style="max-width: 1920px; width: 100\%">
    <div style="position: relative;
        padding-bottom: 56.25\%; height: 0; overflow: hidden;">
      <iframe width="1920" height="1080"
        src="https://www.youtube-nocookie.com/embed/#1"
        title="YouTube: #2" allowfullscreen=""
        style="border:none; position: absolute; top: 0; left: 0;
          right: 0; bottom: 0; height: 100\%; max-width: 100\%;"
        allow="accelerometer; autoplay; clipboard-write;
          encrypted-media; gyroscope; picture-in-picture"/>
    </div>
  </div>}
\iflatexml\else
\begin{center}
  Watch \href{https://www.youtube.com/watch?v=#1}{#2}.
\end{center}
\fi}
\youtube{mH0oCDa74tE}
  {Group theory, abstraction, and the 196,883-dimensional monster}
  \end{lstlisting}
  \newcommand{\youtube}[2]{\bmlRawHTML{
    <div style="max-width: 1920px; width: 100\%">
      <div style="position: relative;
          padding-bottom: 56.25\%; height: 0; overflow: hidden;">
        <iframe width="1920" height="1080"
          src="https://www.youtube-nocookie.com/embed/#1"
          title="YouTube: #2" allowfullscreen=""
          style="border:none; position: absolute; top: 0; left: 0;
            right: 0; bottom: 0; height: 100\%; max-width: 100\%;"
          allow="accelerometer; autoplay; clipboard-write;
            encrypted-media; gyroscope; picture-in-picture"/>
      </div>
    </div>}
  \iflatexml\else
  \begin{center}
    Watch \href{https://www.youtube.com/watch?v=#1}{#2}.
  \end{center}
  \fi}
  \youtube{mH0oCDa74tE}
    {Group theory, abstraction, and the 196,883-dimensional monster}
  \caption{Demonstration of \iflatexml\ltxinline|\bmlRawHTML|\else\ltxinline|\\bmlRawHTML|\fi{} within \iflatexml\ltxinline|\newcommand|\else\ltxinline|\\newcommand|\fi{} with a video from \href{https://www.youtube.com/3blue1brown}{3Blue1Brown}.}
  \label{fig:youtube-ducks}
\end{figure}

\subsection{Interspersing \LaTeX{} and \HTML{} (beta)}

The command \ltxinline|\bmlHTMLEnvironment{tag}| defines an environment \ltxinline|\begin{h:tag} ... \end{h:tag}| which wraps the content between \lstinline[language=html,frame=none]|<tag> ... </tag>|. One can also add attributes as optional arguments \ltxinline|\begin{h:tag}[attr1=val1,attr2=val2]|.

Use \ltxinline|\bmlHTMLInlineEnvironment{tag}| for tags that can only contain `phrasing' content, for instance they should not contain \htmlinline|<p>| paragraphs.

\begin{figure}
  \bmlHTMLEnvironment{details}
  \bmlHTMLInlineEnvironment{summary}
  \begin{h:details}[style={text-align: left; width: 100\%},open]
    \begin{h:summary}
      \textbf{Code for \lstinline[language=html,frame=none]|<details>|.}
    \end{h:summary}

    Completing the quine is left as an exercise for the reader.
    \begin{lstlisting}[style=bookml]
\bmlHTMLEnvironment{details}
\bmlHTMLInlineEnvironment{summary}
\begin{h:details}[style={text-align: left; width: 100\%},open]
  \begin{h:summary}
    \textbf{Code for \lstinline[language=html,frame=none]|<details>|.}
  \end{h:summary}

  Completing the quine is left as an exercise for the reader.
\end{h:details}
    \end{lstlisting}
  \end{h:details}
  \caption{Implementation of the \htmlinline|<details>| tag.}
  \label{fig:details}
\end{figure}

\subsection{...and everything from \texttt{latexml.sty}}
By using \ltxinline|\usepackage{bookml/bookml}|, you will also import \texttt{latexml.sty}, making several \LaTeXML{}-related commands available, for instance \ltxinline|\lxBeginTableHead| for marking table headers. Please read the \href{https://github.com/brucemiller/LaTeXML/blob/81f2f004bdaa6860d673382837c4d4e3951a602e/lib/LaTeXML/texmf/latexml.sty}{source of \texttt{latexml.sty}} to learn what is included.

\end{document}
