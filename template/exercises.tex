\documentclass[oneside,11pt]{amsart}

\usepackage[utf8]{inputenc}
\usepackage[T1]{fontenc}
\usepackage[british]{babel}
\usepackage[pdfusetitle]{hyperref}
\usepackage{lmodern}

\usepackage{bookml/bookml}

% alternative formats
% we use \jobname to distinguish 'exercises' from 'exercises+solutions' automatically
\usepackage{geometry}
\bmlAltFormat{\jobname.pdf}{PDF (serif)}
\bmlAltFormat{\jobname-sans.pdf}{PDF (sans serif)}
\bmlAltFormat{\jobname-sans-large.pdf}{PDF (sans, large)}

% same trick as template.tex
\ifcsname bmlCrop\endcsname
\usepackage{crop}
\fi

% the package comment offers an easy way to conditionally hide paragraphs
\usepackage{comment}
\ifcsname printSolutions\endcsname
\includecomment{solution}
\else
\excludecomment{solution}
\fi

% solution of third exercise
\ifcsname printSolutions\endcsname
  \newenvironment*{sol}
    {\<DETAILS>\<SUMMARY>\noindent\textbf{Solution.}\</SUMMARY>}
    {\</DETAILS>}
\else
  \excludecomment{sol}
\fi

% to pretty print the code of one of the exercises
\usepackage{listings}
\usepackage[dvipsnames]{xcolor}
\colorlet{WCAGGreen}{OliveGreen!80!black} % a bit darker to get sufficient contrast
\lstset{basicstyle={\small\ttfamily},language=[LaTeX]TeX,%
  texcsstyle=*{\color{blue}\bfseries},%
  moretexcs={ifcsname,excludecomment},%
  stringstyle={\color{red}},%
  frame=none,%
  xleftmargin=1em,xrightmargin=1em}

\title{BookML example exercises}
\author{Vincenzo Mantova}
\date{21th March 2023}

\begin{document}

\maketitle

\begin{itemize}
  \item Run \texttt{make} in this folder, then modify \texttt{section1.tex}. Which files will be regenerated the next time you run \texttt{make}? In which order?

\begin{solution}
  \noindent\textbf{Solution.} \texttt{template.pdf}, \texttt{template.xml}, \texttt{template-sans.pdf}, \texttt{template-sans-large.pdf}, \texttt{template/index.html}, \texttt{template.zip}, \texttt{SCORM.template.zip}.
\end{solution}
  \item Modify \texttt{Makefile} so that \texttt{template.zip} (and related SCORM package) uses a single page, rather than being split by section.

\begin{solution}
  \noindent\textbf{Solution.} Add the line \texttt{SCORM.template.zip template.zip: SPLITAT=} at the bottom of \texttt{Makefile}.
\end{solution}

  \item Create an environment \texttt{sol} that creates a foldable `Solutions' block in the HTML output, while it is also omitted unless \texttt{\textbackslash{}printSolutions} is defined. Namely, something like:

\<DETAILS>\<SUMMARY>\noindent\textbf{Sample solution.}\</SUMMARY>
  Solution text goes here.
\</DETAILS>

\begin{sol}
  Use the following code in the preamble
  \begin{lstlisting}
\ifcsname printSolutions\endcsname
  \newenvironment*{sol}
    {\<DETAILS>\<SUMMARY>\noindent\textbf{Solution.}\</SUMMARY>}
    {\</DETAILS>}
\else
  \excludecomment{sol}
\fi
  \end{lstlisting}
  The environment \texttt{sol} can now be used for solutions.
\end{sol}
\end{itemize}

\end{document}
