\documentclass[a4paper,british]{article}

\usepackage[T1]{fontenc}

\usepackage[british]{babel}
\usepackage{bookml/bookml}

% Google Search Console verification file
\iflatexml
\lxRequireResource[type=application/octet-stream]{googlee1709314ff1666e6.html}
\fi

% better fonts
\usepackage{lmodern}
\usepackage{microtype}

\bmlAltFormat{docs.pdf}{PDF (serif)}
\bmlAltFormat{docs-sans.pdf}{PDF (sans serif)}
\bmlAltFormat{docs-sans-large.pdf}{PDF (sans, large)}
\ifcsname bmlCrop\endcsname
\usepackage{crop}
\fi

\usepackage[pdfusetitle,colorlinks]{hyperref}
\usepackage{ifpdf}
\ifpdf % make build reproducible
\pdfinfoomitdate=1
\pdftrailerid{}
\pdfsuppressptexinfo=-1
\hypersetup{pdfinfo={Creator={},Producer={}}}
\hyperbaseurl{https://vlmantova.github.io/bookml/}
\fi

\hypersetup{pdfsubject={BookML is a fully automated solution for the production of accessible HTML content straight from LaTeX, based on LaTeXML for the widest LaTeX compatibility and bookdown for a modern and accessible look. Integration with Overleaf is provided via a GitHub action. Outputs are also packaged as SCORM for ease of use in higher education. Created by and maintained for maths lecturers at the University of Leeds.}}

\usepackage[dvipsnames]{xcolor}
\usepackage{geometry}
\setlength{\parindent}{0pt}
\setlength{\parskip}{0.6em}

\usepackage{tabularx}
% \renewcommand\tabularxcolumn[1]{m{#1}}% for vertical centering text in X columns
\renewcommand\tabularxcolumn[1]{>{\raggedright\arraybackslash}p{#1}}% for vertical centering text in X columns

\usepackage[all]{xy}

\iflatexml
\else
\usepackage{tikz}
\usetikzlibrary{ducks}
\usetikzlibrary{animations}
\usepackage{animate}
\fi
\def\tikzname{Ti\emph{k}Z}

\bmlImageEnvironment{animateinline}
\bmlImageEnvironment{tikzpicture}

\usepackage{amssymb} % framed needs \blacktriangleright
\usepackage{framed}
\iflatexml
  \newenvironment{foldedframe}[2][]{%
    \subsection*[#2]{}\addcontentsline{toc}{subsection}{#2}\<DETAILS #1>\<SUMMARY>\textbf{#2}\</SUMMARY>}%
  {\</DETAILS>}
\else
  % if compiling to PDF, emit {titled-frame} using the framed package
  \newenvironment{foldedframe}[2][]{\addcontentsline{toc}{subsection}{#2}\bgroup\colorlet{TFFrameColor}{SpringGreen}\colorlet{TFTitleColor}{black}\begin{titled-frame}{#2}}{\end{titled-frame}\egroup}
\fi

\newcommand{\youtube}[2]{\bmlRawHTML{<iframe src="https://www.youtube-nocookie.com/embed/#1" title="#2" frameborder="0" allow="picture-in-picture; web-share" referrerpolicy="strict-origin-when-cross-origin" allowfullscreen="" style="width:100\%;max-width:1920px;aspect-ratio:16/9"></iframe>}}

\usepackage{listings}
\colorlet{WCAGGreen}{OliveGreen!80!black} % a bit darker to get sufficient contrast

\usepackage{listings}
\lstset{basicstyle={\small\ttfamily},
  keywordstyle={\color{blue}\bfseries},
  keywordstyle=[2]{\color{MidnightBlue}\bfseries},
  stringstyle={\color{red}},
  commentstyle={\color{OliveGreen}},
  frame={single},frameround={tttt},rulecolor={\color{SpringGreen}},
  upquote=true
}

\lstdefinelanguage{LaTeX}[LaTeX]{TeX}{
  moretexcs={chapter,text,RequirePackage,includegraphics,bf}
}

\lstdefinestyle{latexml}{language=LaTeX,
  texcsstyle=*{\color{blue}\bfseries},
  texcsstyle=*[2]{\color{MidnightBlue}\bfseries},
  moretexcs=[2]{LaTeXML,iflatexml,lxAddClass,lxWithClass,lxBeginTableHead,lxEndTableHead,
    lxFcn,lxID,lxPunct,lxContextTOC,lxNavbar,lxHeader,lxFooter,ldsHTML,bmlRawHTML,
    bmlImageEnvironment,bmlDescription,bmlHTMLEnvironment,bmlHTMLInlineEnvironment,bmlDisableMathJax,<,}
}

\lstdefinestyle{bookml}{language=[LaTeX]TeX,
  texcsstyle=*{\color{blue}\bfseries},
  texcsstyle=*[2]{\color{MidnightBlue}\bfseries},
  moretexcs={xymatrix,ltxinline,href,RequirePackage},
  moretexcs=[2]{LaTeXML,iflatexml,lxAddClass,lxWithClass,lxBeginTableHead,lxEndTableHead,lxFcn,lxID,lxPunct,lxContextTOC,lxNavbar,lxHeader,lxFooter,bmlImageEnvironment,bmlHTMLEnvironment,bmlRawHTML,bmlDescription,bmlPlusClass,bmlHTMLInlineEnvironment,bmlAltFormat}
}

\lstdefinestyle{yaml}{
  basicstyle={\small\ttfamily\color{blue}},
  comment=[l]{:},
  commentstyle={\color{blue!50!black}},
}

\def\ltxinline{\lstinline[style=bookml,frame=none]}
\def\htmlinline{\lstinline[language=html,frame=none]}
\def\cmdinline{\lstinline[language=bash,frame=none]}
\def\makeinline{\lstinline[language=make,frame=none]}
\def\yamlinline{\lstinline[style=yaml,frame=none]}

\usepackage{amsmath}
\usepackage{cleveref}

\title{\texorpdfstring{BookML\@: \LaTeX{} to \HTML, made easy(ish). \\ Powered by \href{https://dlmf.nist.gov/LaTeXML/}{\LaTeXML}.}{BookML: LaTeX to HTML, made easy(ish). Powered by LaTeXML.}}
\author{Vincenzo Mantova}

\date{14\textsuperscript{th} January 2026}

\begin{document}

\maketitle

\begin{abstract}
  BookML is a fully automated solution for the production of accessible \HTML{} content straight from \LaTeX{}, based on \href{https://dlmf.nist.gov/LaTeXML/}{\LaTeXML} for the widest \LaTeX{} compatibility and \href{https://bookdown.org/home/}{bookdown} for a modern and accessible look. Integration with \href{https://www.overleaf.com/}{Overleaf} is provided via a \href{https://github.com/marketplace/actions/compile-with-bookml}{GitHub action}. Outputs are also packaged as \href{https://scorm.com/}{SCORM} for ease of use in higher education. Created by and maintained for maths lecturers at the \href{https://eps.leeds.ac.uk/maths}{University of Leeds}.
\end{abstract}

\begin{center}
  Downloads and sources available on \href{https://github.com/vlmantova/bookml/}{GitHub}.
  \bmlRawHTML{
    <br/>
    <a href="https://github.com/vlmantova/bookml/releases/latest"><img alt="Latest release" src="https://img.shields.io/github/v/release/vlmantova/bookml?logo=github\&amp;display\_name=release"/></a>
    <img alt="Download stats" src="https://img.shields.io/github/downloads/vlmantova/bookml/total?logo=github" class="noepub"/>
    <a href="https://github.com/vlmantova/bookml/stargazers"><img alt="Stargazers on GitHub" src="https://img.shields.io/github/stars/vlmantova/bookml?logo=github"/></a>
  }

  Formats: \href{https://vlmantova.github.io/bookml/}{GitBook} (html), \href{index.plain.html}{plain} with Latin Modern (html), \href{https://vlmantova.github.io/bookml/docs.pdf}{PDF}, \href{https://github.com/vlmantova/bookml/blob/docs/docs.tex}{\LaTeX{} source}.

  For Leeds-specific instructions see the \href{https://vlmantova.github.io/bookmlleeds/}{Leeds BookML guide}.
\end{center}

BookML is a small package to help you convert \LaTeX{} documents into accessible \HTML{} very similar to the one produced by \href{https://bookdown.org/home/}{bookdown}. The conversion is done by \href{https://dlmf.nist.gov/LaTeXML/}{\LaTeXML{}}, which understands a wide collection of \LaTeX{} packages and can deal with all sorts weird constructs. To see plain(ish) \LaTeXML{} in action, just open any \href{https://info.arxiv.org/about/accessible_HTML.html}{arXiv \HTML} preprint. For examples of BookML documents, see:

\begin{itemize}
  \item \href{https://vlmantova.github.io/bookml/}{This manual}!
  \item \href{https://vlmantova.github.io/bookmlleeds/}{The Leeds BookML guide}.
  \item \href{https://forallx.openlogicproject.org/html/}{forall $x$: Calgary} by Richard Zach.
  \item \href{https://www.maths.lancs.ac.uk/~grabowsj/rtaca/}{Representation Theory. A Categorical Approach} by Jan E.\ Grabowski.
  \item \href{https://www.ime.usp.br/~leorolla/probabilidade/}{Probabilidade} by Leonardo T.\ Rolla and Bernando N.B.\ de Lima.
  \item Some resources of \href{https://rea-mecc.github.io/acervo.html}{REA-MECC}.
\end{itemize}

BookML key features are:
\begin{itemize}
  \item simple installation: BookML is just a zip file you unpack in the folder containing your \texttt{.tex} files (admittedly a lie: you must install \LaTeXML{} first, see \hyperref[get-started-on-your-device]{Getting started on your device});
  \item accessible and mobile friendly output based on the popular \href{https://bookdown.org/home/}{bookdown} project, including font selection and dark mode, tweaked to meet the \href{https://www.w3.org/TR/WCAG21/}{Web Content Accessibility Guidelines 2.1} level AA;
  \item fully automated (re-)compilation based on which files have changed on disk, powered by \href{https://www.gnu.org/software/make/}{GNU Make}: running a single \cmdinline{make} will recompile the files which need updating;
  \item high quality conversion of external EPS and PDF figures to SVG via \href{https://dvisvgm.de}{dvisvgm}, rather than ImageMagick used by LaTeXML;
  \item transparent generation of SVG images from \href{https://tikz.net/}{\tikzname} pictures, \href{https://ctan.org/pkg/animate}{animate} animations, \Xy-matrices, and virtually any other picture-like environment, for when \LaTeXML{} struggles with them;
  \item easy insertion of alternative text for images;
  \item arbitrary \HTML{} content in \LaTeX{}, for instance to create foldable paragraphs;
  \item declare and include alternative PDF versions, which are also recompiled automatically (for instance sans serif, large print);
  \item outputs are automatically packaged into zip files and \href{https://scorm.com/}{SCORM packages}, which are supported natively by most Learning Management Systems.
\end{itemize}

\tableofcontents

\section{Getting started}\label{get-started}

\subsection{On your device}\label{get-started-on-your-device}

\begin{enumerate}
  \item Install the bare minimum prerequisites: \href{https://dlmf.nist.gov/LaTeXML/get.html}{LaTeXML} (minimum 0.8.7, recommended 0.8.8 or later) and \href{https://www.gnu.org/software/make/}{GNU Make} (minimum 3.81, but later versions are considerably faster), on top of a working \TeX{} installation, such as \TeX{} Live or MiK\TeX{}.
  \item Unpack the latest \href{https://github.com/vlmantova/bookml/releases/latest/download/release.zip}{BookML release} and put the \texttt{bookml} folder next to your \texttt{.tex} files. You can also start with a \href{https://github.com/vlmantova/bookml/releases/latest/download/template.zip}{minimal template}.
  \item \textbf{First time only:} move \texttt{bookml/GNUmakefile} one folder up. It must be in the same folder as your main \texttt{.tex} files.
  \item Now open a terminal in the folder containing \texttt{GNUmakefile} and run \cmdinline{make detect} (use \cmdinline{gmake detect} on Windows).
\end{enumerate}

When all goes well, you should see an output like the following:

\begin{tabularx}{\textwidth}[h]{>{\color{MidnightBlue}\tt}r@{\texttt{\ }}>{\tt}X}
  Main files:    & \textcolor{OliveGreen}{template.tex}                                          \\
  BookML:        & \textcolor{OliveGreen}{\BookMLversion{} OK}                                   \\
  GNU Make:      & \textcolor{OliveGreen}{4.4.1 OK}                                              \\
  TeX:           & \textcolor{OliveGreen}{MiKTeX 25.4 OK}                                        \\
  perl:          & \textcolor{OliveGreen}{v5.42.0 OK}                                            \\
  LaTeXML:       & \textcolor{OliveGreen}{0.8.8 OK}                                              \\
  Image::Magick: & \textcolor{red}{NOT FOUND} (required for any image handling)                  \\
  Ghostscript:   & \textcolor{OliveGreen}{9.25 OK} (required for BookML images, EPS to SVG; may be required for PDF to SVG) \\
  mutool:        & \textcolor{OliveGreen}{1.23.0 OK} (required for PDF to SVG if using PDFTOSVG\_CONVERTER=mutool (current value `mutool')) \\
  dvisvgm:       & \textcolor{OliveGreen}{3.4.3 OK} (required for BookML images, EPS to SVG, PDF to SVG if using PDFTOSVG\_CONVERTER=dvisvgm (current value `mutool')) \\
  dvisvgm/libgs: & \textcolor{OliveGreen}{9.25 OK} (required for BookML images, EPS to SVG, PDF to SVG if using PDFTOSVG\_CONVERTER=dvisvgm (current value `mutool'))  \\
  latexmk:       & \textcolor{OliveGreen}{4.87 OK}                                               \\
  texfot:        & \textcolor{OliveGreen}{1.48 OK} (optional, for hiding some LaTeX messages)    \\
  preview.sty:   & \textcolor{OliveGreen}{14.0.6 OK} (required for BookML images)                \\
  zip:           & \textcolor{OliveGreen}{3.1b OK}                                               \\
  curl:          & \textcolor{OliveGreen}{8.13.0 OK} (required for updating with `make update')
\end{tabularx}

The first line shows which files BookML will try to compile: every \texttt{.tex} file in the current folder that contains the string \ltxinline|\documentclass| (even if commented out!). The rest informs you on which software you may need to install or update. \textbf{Not everything is necessary:} the output of \cmdinline{make detect} will tell you what functionality requires a particular program. The \href{https://vlmantova.github.io/bookmlleeds/}{Leeds BookML guide} has detailed installation instructions for various platforms, many specific to the University of Leeds.

Finally, run \cmdinline{make} (or \cmdinline{gmake} on Windows). BookML will compile each `main file' to a zip file and a SCORM package. You can view the results in the folder \texttt{auxdir/html}.

\textbf{Tip:} the first time you run BookML against a new document, add an early \ltxinline|\end{document}|, say after one or two pages, to check whether \LaTeXML{} is handling well your selection of packages and macros.

\subsection{Overleaf}\label{get-started-on-overleaf}
If you have a paid Overleaf plan, you can also compile your Overleaf project on GitHub's servers.
\begin{enumerate}
  \item Open your Overleaf \href{https://www.overleaf.com/user/settings}{account settings} and \href{https://docs.overleaf.com/integrations-and-add-ons/git-integration-and-github-synchronization/github-synchronization#linking-your-overleaf-account-to-your-github-account}{link your GitHub account} from the `GitHub Sync' entry.
  \item Open an existing or new Overleaf project and start \href{https://docs.overleaf.com/integrations-and-add-ons/git-integration-and-github-synchronization/github-synchronization#creating-a-new-github-repository-from-an-overleaf-project}{synchronizing it with a new GitHub repository}. The process is smoother when \textbf{the project name has no spaces}. \textbf{Note:} I recommend you immediately visit the new repository and click the `Watch' icon (next to `Edit Pins', `Fork', `Star').
  \item Finally, add the file \texttt{.github/workflows/bookml.yaml} to the project with the following content:
    \begin{lstlisting}[style=yaml,name=bookml.yaml]
on: push
jobs:
  build:
    runs-on: ubuntu-latest
    permissions:
      contents: write
    steps:
      - name: Compile with BookML
        uses: vlmantova/bookml-action@v1
    \end{lstlisting}
    Now synchronize the project with GitHub. \textbf{First time only:} GitHub will ask you to grant permissions to Overleaf to run workflows. Say yes!
  \item GitHub should start compiling all \texttt{.tex} files at the top folder of your Overleaf project that contain the string \ltxinline|\documentclass| (even if commented out!). The work is done by the BookML Docker image, which at the moment contains a full copy of \TeX{} Live 2021.
  \item If you `watched' the repository as recommended in the previous steps, you should receive an email in 2-3 minutes with the download links of all the outputs compiled by BookML and any relevant error messages. If you do not watch the repository, you must visit the repository page yourself and look for new entries under `Releases' (below `About') or open the latest workflow run under the `Action' tab.
  \item Every time you want to compile a new version of your Overleaf project, just push the new changes to GitHub.
\end{enumerate}

\subsection{GitHub}\label{get-started-on-github}
\begin{enumerate}
  \item Simply add the file \texttt{.github/workflows/bookml.yaml} to your repositories with the following content:
    \begin{lstlisting}[style=yaml,name=bookml.yaml]
on: push
jobs:
  build:
    runs-on: ubuntu-latest
    permissions:
      contents: write
    steps:
      - name: Compile with BookML
        uses: vlmantova/bookml-action@v1
    \end{lstlisting}
    I also recommend that you watch the repository (this is usually automatic when you create a repository; see the `Watch' icon on the repository page, next to `Edit Pins', `Fork', `Star').
  \item GitHub should start compiling all \texttt{.tex} files at the top folder of your repository that contain the string \ltxinline|\documentclass| (even if commented out!). The work is done by the BookML Docker image, which at the moment contains a full copy of \TeX{} Live 2021.
  \item If you are watching the repository as recommended in the previous steps, you should receive an email in 2-3 minutes with the download links of all the outputs compiled by BookML and any relevant error messages. If you do not watch the repository, you must visit the repository page yourself and look for new entries under `Releases' (below `About') or open the latest workflow run under the `Action' tab.
  \item Every push will start a new compile. If you push frequently, consider tweaking the action to run only when pushing to \href{https://docs.github.com/en/actions/reference/workflows-and-actions/events-that-trigger-workflows#running-your-workflow-only-when-a-push-to-specific-branches-occurs}{specific branches} or \href{https://docs.github.com/en/actions/reference/workflows-and-actions/events-that-trigger-workflows#running-your-workflow-only-when-a-push-of-specific-tags-occurs}{tags}.
\end{enumerate}

If you `watched' the repository as recommended in the previous steps, you should receive an email in 2-3 minutes with the download links of all the outputs compiled by BookML and any relevant error messages. If you do not watch the repository, you must visit the repository page yourself and look for new entries under `Releases' (below `About') or open the latest workflow run under the `Action' tab.

\subsection{Docker/Podman}\label{get-started-on-docker-podman}
BookML is also available as a Docker image, which automatically compile all \texttt{.tex} files at the top of the \texttt{/source} folder containing the string \ltxinline|\documentclass| (even if commented out!). The image is identical to the one used for Overleaf, and so it contains a full copy of \TeX{} Live 2021.

You can run the Docker image by opening a terminal in the folder containing the \texttt{.tex} files and running
\begin{lstlisting}[language=bash]
docker run --rm -i -t -v.:/source ghcr.io/vlmantova/bookml
\end{lstlisting}
Additional arguments are passed to \texttt{make}, for instance
\begin{lstlisting}[language=bash]
docker run --rm -i -t -v.:/source ghcr.io/vlmantova/bookml detect
\end{lstlisting}
will show an output similar to the one described in \hyperref[get-started-on-your-device]{Getting started on your device}.

\section{Examples and tips}\label{examples-and-tips}

While many \texttt{.tex} files will be fine with no changes, you may want to tweak the output further by invoking the \ltxinline|\usepackage{bookml/bookml}| or by adding settings to \texttt{GNUmakefile}. When using Overleaf, that means you also need to upload the \texttt{bookml/} folder. Below are some notable examples. For the full list of options and commands, see the \hyperref[reference-manual]{Reference manual}.

\begin{foldedframe}[open=""]{Alternative text}\label{alternative-text}
  For \ltxinline|\includegraphics|, just add the option \ltxinline|alt={description of the image}|, as in
  \begin{lstlisting}[style=bookml]
\includegraphics[alt={Computation of ...}]{figure1}
  \end{lstlisting}

For other images such as \tikzname{} pictures, add \ltxinline|\bmlDescription{text}| \textbf{right after} the image. \textbf{Do not leave any space between the picture and }\ltxinline|\bmlDescription|\textbf{!} Please remember that sighted users can also benefit from descriptions. Consider using \ltxinline|\begin{figure}| and \ltxinline|\caption{}| instead.
\end{foldedframe}

\begin{foldedframe}[open=""]{Animations}\label{animations}
  Both \texttt{animate} and \tikzname{} can produce animations, but \LaTeXML{} does not support this at the moment. BookML can transparently convert the animations to SVG using \ltxinline|\bmlImageEnvironment|. For instance, add the following to the preamble:
  \begin{lstlisting}[style=bookml]
\bmlImageEnvironment{animateinline}
  \end{lstlisting}

  Then all animations created with \ltxinline|animateinline| from the \texttt{animate} package will be converted faithfully.

  \begin{center}
    \begin{animateinline}[
      alttext=none,loop,controls,nomouse,poster=first,autopause,
      begin={\begin{tikzpicture}
        \clip (-1,-1) rectangle (3,2.5);},
        end={\end{tikzpicture}}]{2}
      \multiframe{3}{nTime=.3+.3}
        {\duck[tophat,scale=\nTime,xshift=\nTime*2]}
    \end{animateinline}
  \end{center}

  For the occasional \tikzname{} figure, adding \ltxinline|\begin{bmlimage}| around it will suffice.
  \begin{lstlisting}[style=bookml]
\begin{bmlimage}\begin{tikzpicture}...\end{tikzpicture}\end{bmlimage}
  \end{lstlisting}
  \begin{center}
    \begin{bmlimage}
      \begin{tikzpicture}
        \clip (1,-0.5) rectangle (8,2.5);
        \begin{scope}[animate = {myself:xshift = {0s = "-2cm", 6s = "10cm", repeats}, myself:rotate = {0s = "0", 6s = "2160", origin = {(1,1)}, repeats}}]
          \duck[tophat]
        \end{scope}
      \end{tikzpicture}
    \end{bmlimage}
  \end{center}
\end{foldedframe}

\begin{foldedframe}[open=""]{Difficult pictures}\label{difficult-pictures}
  If the occasional picture takes too long to compile, or it comes out wrong, wrap it in \ltxinline|\begin{bmlimage}|.
  \[ \text{straight \Xy-matrix: }
  \xymatrix{
    A \ar[rd] \ar^\phi[r] & B \\
                          & C } \]
  \[ \text{\Xy-matrix in \texttt{bmlimage}: }\begin{bmlimage}
  \xymatrix{
    A \ar[rd] \ar^\phi[r] & B \\
                          & C }
  \end{bmlimage} \]
  \begin{lstlisting}[style=bookml]
\[ \text{\Xy-matrix in \texttt{bmlimage}: }\begin{bmlimage}
\xymatrix{
  A \ar[rd] \ar^\phi[r] & B \\
                        & C }
\end{bmlimage} \]
  \end{lstlisting}

  To deal with \textbf{every} \tikzname{} picture automatically, use \ltxinline|\bmlImageEnvironment{}|. The following code shows how to even \textbf{not load \tikzname{} at all}, so that \LaTeXML{} can run through the preamble faster, saving potentially up to 1 minute.
  \begin{lstlisting}[style=bookml]
\bmlImageEnvironment{tikzpicture}
\iflatexml
\else
\usepackage{tikz}
% ... ALL of the TikZ-related preamble code here
\fi
% No TikZ commands after this point!
  \end{lstlisting}
\end{foldedframe}

\begin{foldedframe}[open=""]{Videos}\label{videos}
  You can inject arbitrary \HTML{} using \ltxinline|\bmlRawHTML{HTML here}| or \ltxinline|\<div>TeX here\</div>|. That is enough to insert videos, for example by typing the YouTube embed code. Note that characters must be escaped, for instance replace \ltxinline|%| with \ltxinline|\%|, and you must follow the \XML{} syntax, so an attribute \htmlinline|<iframe allowfullscreen>| will need to be written as \htmlinline|<iframe allowfullscreen="">|.
  \begin{lstlisting}[style=bookml]
\bmlRawHTML{<iframe allowfullscreen=""></iframe>}
  \end{lstlisting}

  I advise you to remove \htmlinline|width| and \htmlinline|height| and replace them with a suitable \htmlinline|style|, such as:
  \begin{lstlisting}[style=bookml]
\bmlRawHTML{<iframe style="width:100\%;
  max-width:1920px;aspect-ratio:16/9"
  src="https://www.youtube-nocookie.com/..." ...></iframe>}
  \end{lstlisting}

  You can even create a macro:
  \begin{lstlisting}[style=bookml]
\newcommand{\youtube}[2]{\bmlRawHTML{<iframe
    src="https://www.youtube-nocookie.com/embed/#1"
    title="#2" style=...></iframe>}}
  \end{lstlisting}

  \youtube{pTqBXcUPDug?si=tfkXijIyz4dgB38Z}{SCORM upload to Blackboard Learn Ultra}
  % \bmlRawHTML{<iframe src="https://www.youtube-nocookie.com/embed/pTqBXcUPDug?si=tfkXijIyz4dgB38Z" title="SCORM upload to Blackboard Learn Ultra" frameborder="0" allow="picture-in-picture; web-share" referrerpolicy="strict-origin-when-cross-origin" allowfullscreen="" style="width:100\%;max-width:1920px;aspect-ratio:16/9"></iframe>}
\end{foldedframe}

\begin{foldedframe}[open=""]{Foldable environments}\label{foldable-environments}
  If you want to hide a proof, a solution, or some additional details, you can use the \htmlinline|<details>| \HTML{} tag, as follows:

  \begin{lstlisting}[style=bookml]
\<details>
  \<summary>\textbf{Solution.}\</summary>
  ...details of the solutions...
\</details>
    \end{lstlisting}

  \<details>
    \<summary>\textbf{Solution.}\</summary>
    ...details of the solutions...
  \</details>

  I suggest creating an environment, for instance:
  \begin{lstlisting}[style=bookml]
\iflatexml % for the HTML
\newenvironment{solution}
  {\<details>\<summary>Solution.\</summary>}
  {\</details>}
\else      % for the PDF
\newenvironment{solution}{\begin{proof}[Solution]}{\end{proof}}
\fi
  \end{lstlisting}

  The \HTML{} tags must be written in \XML{} syntax (so \ltxinline|\<br/>| rather than \ltxinline|\<br>|). See \hyperref[reference-manual]{Reference manual} for more details.
\end{foldedframe}

\begin{foldedframe}{Customize header and footer (e.g.\ for copyright notice)}\label{headers-and-footers}
  Use the environment \ltxinline|lxFooter|.
  \begin{lstlisting}[style=bookml]
\begin{lxFooter}
  Copyright \copyright{} 2026 Vincenzo Mantova, University of Leeds.
\end{lxFooter}
    \end{lstlisting}
  The content will appear on each page, but it will be ignored in the PDF. There is also a \ltxinline|lxHeader| environment which can be used with plain and no styles.
\end{foldedframe}

\begin{foldedframe}{Split by chapter instead of section, or not split at all}\label{split-by-chapter}
  Add \makeinline|SPLITAT=chapter| to \texttt{GNUmakefile} to split the output by chapters (you can use part, chapter, section\ldots). Write \makeinline|SPLITAT=| to output a single page.

  You can also specify different splitting strategies for different files by writing:
  \begin{lstlisting}
$(AUX_DIR)/html/lecturenotes/index.html: SPLITAT=chapter
$(AUX_DIR)/html/problemsheet1/index.html: SPLITAT=
\end{lstlisting}
  to the end of \texttt{GNUmakefile}. Be careful to specify the splitting on the \texttt{index.html} files rather than the final zip or SCORM target. See \hyperref[reference-manual]{Reference manual} for more details.

  By default, all files are split by section.
\end{foldedframe}

\begin{foldedframe}{Change MathJax version}\label{mathjax-versions}
  The current default for BookML is MathJax 3, but MathJax 4 already seems to work well for most documents. If you want to jump ahead, use
  \begin{lstlisting}[style=bookml]
\usepackage[mathjax=4]{bookml/bookml}
  \end{lstlisting}
\end{foldedframe}

\begin{foldedframe}{Disable MathJax for an equation}\label{disable-mathjax}
  Add \ltxinline|\bmlDisableMathJax| within the equation. For environments generating multiple equations, such as \ltxinline|\begin{align*}|, the command will only have effect on the row it appears. Please review the output in Firefox, Safari, and Chrome/Edge as it is likely to look different across different browsers.
\end{foldedframe}

\begin{foldedframe}{Alternative formats}\label{alternative-formats}
  By default, BookML includes the PDF in the final output. Additional versions can be added by calling for instance
  \begin{lstlisting}[style=bookml]
\bmlAltFormat{docs.large.pdf}{PDF (large print)}
  \end{lstlisting}
  Then BookML will compile \texttt{docs.large.tex} to PDF and include it in the \HTML{} output. The file will appear in the `Downloads' menu of the bookdown style.

  The inclusion of the PDF can be stopped by specifying an empty name:
  \begin{lstlisting}[style=bookml]
\bmlAltFormat{docs.pdf}{}
  \end{lstlisting}

  Please note that \texttt{docs.large.tex} must \emph{not} contain the string \ltxinline|\documentclass| or it will be compiled by itself into separate \HTML{}, zip, and SCORM.

  See \texttt{template.tex}, \texttt{template-sans.tex}, \texttt{template-sans-large.tex} for an example.
\end{foldedframe}

\begin{foldedframe}{Set the SCORM description}\label{scorm-description}
  \begin{lstlisting}[style=bookml]
\usepackage{hyperref}
\hypersetup{pdfsubject={Description of this file}}
  \end{lstlisting}
  will set the SCORM package description so you won't need to type it manually in your LMS.

  Note that the metadata will also be included in the PDF. Use \ltxinline|\iflatexml...\fi| appropriately if you need different information. If the \texttt{pdfsubject} is not present, the abstract will be used as description.
\end{foldedframe}

\begin{foldedframe}{Some PDF and EPS figures look odd or prevent compilation}\label{odd-pdf-eps-figures}
  By default, BookML uses \texttt{dvisvgm} to convert EPS images to SVG, and \texttt{mutool} to convert PDF images to SVG. If you do not want to install \texttt{mutool}, you can try setting \makeinline|PDFTOSVG_CONVERTER=dvisvgm| in \texttt{GNUmakefile}. The output of \cmdinline|make detect| will help you determine if this is feasible on your machine.

  If you cannot get good SVGs from the automatic conversion, you have two possibilities. You can convert the images yourself, and place the resulting images in the same folder and with the same name as the original PDF or EPS files, but extensions \texttt{.svg}, \texttt{.png}, or \texttt{.jpg}. \textbf{You must use }\ltxinline|\includegraphics{figure}|\textbf{ instead of }\ltxinline|\includegraphics{figure.pdf}|\textbf{ for this to work.} You can also disable the automatic conversion altogether with \makeinline|PDFTOSVG_CONVERTER=|, at which point \LaTeXML{} will convert the images to PNG using ImageMagick.
\end{foldedframe}

\begin{foldedframe}{Some images look weird in sepia and dark mode}\label{images-in-sepia-dark-mode}
  BookML changes colours to all the images in sepia and dark mode so that they match the surrounding text. This is generally fine for diagrams and graphs, but can be problematic for photos.

  To prevent a specific image from changing, call \ltxinline|\bmlPlusClass{bml_no_invert}| immediately after the image. You can see the difference on the two pictures below.

  \begin{center}
    \begin{tikzpicture}
      \duck[tophat]
    \end{tikzpicture}
    \begin{tikzpicture}
      \duck[tophat]
    \end{tikzpicture}\bmlPlusClass{bml_no_invert}
  \end{center}

  Note that this only applies to images, meaning \ltxinline|\includegraphics| and environments converted using \ltxinline|bmlimage|, but not \tikzname{} pictures processed by \LaTeXML{}.
\end{foldedframe}

\begin{foldedframe}{Run different code for PDF and for \HTML{}}\label{different-code-for-PDF-and-HTML}
  The conditional \ltxinline|\iflatexml| lets you pass different code to \LaTeXML{} and \LaTeX{}.
  \begin{lstlisting}[style=bookml]
\iflatexml
% code run by LaTeXML only, for the HTML
\else
% code run by LaTeX only, for the PDF
\fi
  \end{lstlisting}
\end{foldedframe}

\begin{foldedframe}{Customise \CSS{} (fonts, colors, etc)}\label{customise-CSS}
  Just add \CSS{} files to the \texttt{bmluser/} folder. See \hyperref[reference-manual]{Reference manual} for how files get included.
\end{foldedframe}

\begin{foldedframe}{Disable the bookdown style}\label{disable-bookdown-style}
  If you do not like the bookdown style and prefer a more plain page, use
  \begin{lstlisting}[style=bookml]
\usepackage[style=plain]{bookml/bookml}
  \end{lstlisting}
\end{foldedframe}

\begin{foldedframe}{Table headers}\label{table-headers}
  Marking table cells, rows, or columns as headers can be very important for accessibility, especially for screen reader users. \LaTeXML{} offers a few rudimentary commands to deal with this. Import the \texttt{bookml/bookml} or the \texttt{latexml} package, and write the table as follows
  \begin{lstlisting}[style=bookml]
\begin{tabularx}{\textwidth}{c|X||c}
  \lxBeginTableHead{} Header 1 & Header 2 & Header 3 \\
  \hline \lxEndTableHead{}
  Content & Content & Content \\
  More content & content & content \\
  \hline
\end{tabularx}
  \end{lstlisting}
  Read the content of \href{https://github.com/brucemiller/LaTeXML/blob/master/lib/LaTeXML/texmf/latexml.sty}{\texttt{latexml.sty}} for more details about which commands are available for tables, and how you can use them.

  \begin{tabularx}{\textwidth}{c|X||c}
    \lxBeginTableHead{} Header 1 & Header 2 & Header 3 \\
    \hline \lxEndTableHead{}
    Content & Content & Content \\
    More content & content & content \\
    \hline
  \end{tabularx}
\end{foldedframe}

\begin{foldedframe}{Unsupported packages or classes}\label{unsupported-package-or-classes}
  \LaTeXML{} supports only so many packages (\href{https://dlmf.nist.gov/LaTeXML/manual/included.bindings/}{full list}), and many only partially. If \LaTeXML{} does not recognise a particular package or class, it is sometimes easy to make it work, but it can also be near impossible.
  \begin{itemize}
    \item If you use your own custom-made package or class to keep your favourite packages and options (so essentially a fancy preamble): change its extension to \texttt{.tex} and use \ltxinline|\input| instead of \ltxinline|\usepackage|.
    \item If the package is producing images: use \ltxinline|bmlimage| as for \tikzname{}.
    \item If \LaTeXML{} supports a similar package: replace it (for instance, use \texttt{actuarialangle} instead of \texttt{lifecon}).
    \item If the package is PDF-specific: use \ltxinline|\iflatexml| and \ltxinline|\newcommand| in the preamble to define macros that do something equivalent, or nothing at all, in \HTML{}. Useful, for instance, for references including page numbers, or setting headers and footers.
    \item If the class is not supported, tell \LaTeXML{} to use a different class:
      \begin{lstlisting}[style=bookml]
\RequirePackage{latexml}
\iflatexml
\documentclass[12pt]{book}
\else
\documentclass[12pt]{memoir}
\fi
      \end{lstlisting}
      Make sure your class options are consistent, for instance make sure you are using the same font size.
    \item If none of the above works, copy the missing macros directly from the package (or class) and add them to your preamble (usually within \ltxinline|\makeatletter| and \ltxinline|\makeatother|). It may work for simple packages, or packages partially supported by \LaTeXML{}.
    \item Last resort, you can tell \LaTeXML{} to attempt reading the entire package or class. To do this, create a file named \texttt{package.sty.ltxml} in the same folder as the \texttt{.tex} file (replace \texttt{package} with the actual package name, and \texttt{sty} with \texttt{cls} if dealing with a class):
    \begin{lstlisting}[language=Perl,name=package.sty.ltxml]
use LaTeXML::Package;
InputDefinitions('package', type => 'sty', noltxml => 1);
1;
      \end{lstlisting}
    This will instruct \LaTeXML{} to read the content of \texttt{package.sty}. This may work wonderfully, or crash miserably.
  \end{itemize}
\end{foldedframe}

\section{Reference manual}\label{reference-manual}
This section follows closely the reference manual on the \href{https://github.com/vlmantova/bookml}{BookML repository}, but may lag behind new versions.

\subsection{Package options}\label{package-options}
BookML can be configured by loading the \texttt{bookml/bookml} package and passing options to it like with any other LaTeX packages, for instance \ltxinline|\usepackage[mathjax=4]{bookml/bookml}|. The following options are available:

\begin{description}
\item[\texttt{style=<\textit{name}>}] Switch style. \texttt{<\textit{name}>} can be \texttt{\textit{gitbook}} (default), which is almost identical to the output of bookdown, \texttt{\textit{plain}}, which is a tweaked version of the normal LaTeXML style, and \texttt{\textit{none}} for the default LaTeXML style with minimal compatibility and bug fixes only.
\item[\texttt{mathjax=<\textit{number}>}] Select which MathJax version to use from 2, 3 (default), 4.
\item[\texttt{nomathjax}] Disable MathJax.
\item[\texttt{imagescale=X.XX}] (Deprecated) Rescale all images generated via LaTeX (using \ltxinline|bmlimage|, see below) by the desired factor. Images are normally sized so that fonts inside the image match the font size of the browser, but there have been cases where BookML is wrong, and images turn out too small or too large. When that happens, tweak \texttt{\textit{imagescale}}, and please report the issue.
\item[\texttt{nohtmlsyntax}] Do not define the command \ltxinline|\<| used for writing \HTML{} tags directly in \TeX{}.
\end{description}

\subsection{Package commands and environments}\label{commands-and-environments}
Loading \ltxinline|\usepackage{bookml/bookml}| makes the following commands available. It also loads the \href{https://github.com/vlmantova/bookml/blob/main/latexml.sty}{latexml package}, which provides additional commands such as \ltxinline|\iflatexml| and \ltxinline|\lxRequireResource|.

\begin{description}
\item[\texttt{\textbackslash{}BookMLversion}] The currently running version of BookML.
\item[\texttt{\textbackslash{}bmlAltFormat[<\textit{opts}>]\{<\textit{file}>\}\{<\textit{label}>\}}] Compile (if necessary) and include \texttt{<\textit{file}>} in the download menu with label \texttt{<\textit{label}>}. An empty label removes the file from the download menu. The optional argument \texttt{<\textit{opts}>} is a key-value list passed internally to \ltxinline|\lxRequireResource| (for instance, use \ltxinline|type=application/octet-stream| if LaTeXML is not able to recognise the MIME type of the file). Only available in the \texttt{\textit{gitbook}} style.
\item[\texttt{\textbackslash{}<}] Open or close an \HTML{} tag, as in \ltxinline|\<span class="example">some \LaTeX{} code\</span>|. The content between the tags is normal \LaTeX{} code. Tags will normally generate an additional \htmlinline|<p>...</p>| tag, unless they can only contain 'phrasing content'. If necessary, the behaviour of each tag can be changed with the \ltxinline|\bmlHTML*Environment| commands. Tags must follow the XML syntax, for instance \ltxinline|\<br>| is not valid: you must write \ltxinline|\<br/>|.
\item[\texttt{\textbackslash{}bmlHTMLEnvironment\{<\textit{tag}>\}}] Introduce or redefine an \HTML{} tag environment, to be used as \ltxinline|\begin{h:tag}[attr1=val1,...]...\end{h:tag}| or as \ltxinline|\<tag>|.
\item[\texttt{\textbackslash{}bmlHTMLInlineEnvironment\{<\textit{tag}>}\}] Introduce or redefine an \HTML{} tag environment which accepts `phrasing content' only.
\item[\texttt{\textbackslash{}bmlRawHTML\{<\textit{html}>}\}] Insert \texttt{\texttt{<\textit{html}>}} directly in the output document, after expanding all the \TeX{} macros. When the command appears in the preamble, \texttt{<\textit{html}>} will be part of the head and will be copied in every output page. \texttt{<\textit{html}>} must be written in valid XML syntax, with either no namespace or the correct namespace for \HTML.
\item[\texttt{\textbackslash{}begin\{bmlimage\}}] The body of this environment is compiled directly into an SVG image via \LaTeX, instead of running through \LaTeXML.
\item[\texttt{\textbackslash{}bmlImageEnvironment\{<\textit{env}>\}}] Compile all environments \texttt{<\textit{env}>} directly into SVG images via \LaTeX, instead of running them through \LaTeXML. The most typical example is \ltxinline|\bmlImageEnvironment{tikzpicture}| for when \LaTeXML{} struggles to process \tikzname{} pictures properly or sufficiently quickly. \textbf{Warning:} this will not work properly when the environments are called implicitly (for instance, the package \texttt{tcolorbox} uses \ltxinline|\begin{tikzpicture}| internally to implement its theorems, and that could results in theorems appearing as images, and in the wrong places).
\item[\texttt{\textbackslash{}bmlDescription\{<\textit{text}>\}}] Attach an alternative text \texttt{<\textit{text}>} to the immediately preceding object. Only useful for images, for instance immediately after \ltxinline|\end{tikzpicture}|. \textbf{Warning:} the command must immediately follow the object; even empty spaces can cause issues.
\item[\texttt{\textbackslash{}bmlPlusClass\{\texttt{<\textit{class}>}\}}] Add the \CSS{} class \texttt{<\textit{class}>} to the immediately preceding object (this complements \ltxinline|\lxAddClass| and \ltxinline|\lxWithClass| provided by the \texttt{latexml} package). \textbf{Warning:} the command must immediately follow the object; even empty spaces can cause issues.
\item[\texttt{\textbackslash{}bmlDisableMathJax}] Disable running MathJax on the current mathematical content. When used in an environment that generates multiple equations, it applies only to the current one.
\end{description}

\subsection{Makefile options}\label{makefile-options}

The build process accepts configuration options via Make variables, using the syntax \makeinline|VARIABLE=value|. The options can be passed in three ways:

\begin{enumerate}
  \item In \texttt{GNUmakefile} before \makeinline|include bookml/bookml.mk|. Each option must appear on its own line, without indentation.
  \item On the command line as \cmdinline{make VARIABLE=value}.
  \item To apply an option to a single output \texttt{file}, write it in \texttt{GNUmakefile} somewhere after \makeinline|include bookml/bookml.mk| as \makeinline|file: VARIABLE=value| on its own line, without indentation. For instance, \makeinline|main.pdf: LATEXMKFLAGS=-pdflua|. \textbf{Warning:} the variable must be applied to the target it affects \textbf{immediately} or it can cause inconsistent results (see for example \makeinline|SPLITAT| below).
\end{enumerate}

Note that changing options will \textbf{not} trigger a recompilation of the files. You will typically need to run \cmdinline|make clean| before recompiling again. For more information about the Makefile syntax and how variables are evaluated, consult the \href{https://www.gnu.org/software/make/manual/}{GNU Make manual}.

The following options are available.

\begin{description}
\item[\texttt{AUX\_DIR}] Location of the directory containing all intermediate files generated during compilation, such as \texttt{.aux} and \texttt{.bbl} files. This option is ignored by the BookML GitHub action. Default \texttt{\textit{auxdir}}.
\item[\texttt{SOURCES}] Space-separated list of \texttt{.tex} files to be compiled. File names with spaces are \textbf{not} supported. Default is the list of \texttt{.tex} files in the current directory that contain the string \ltxinline|\documentclass| (even if appearing in a comment!).
\item[\texttt{FORMATS}] Spaces-separated list of formats to be generated from \texttt{SOURCES}. Recognised formats are \texttt{pdf}, \texttt{scorm}, \texttt{zip}. Default \texttt{\textit{scorm zip}}.
\item[\texttt{SPLITAT}] How to split the \HTML{} output into multiple files (chapter, section, subsection, subsubsection). Set to empty to disable splitting. See the \texttt{latexmlpost} manual, \cmdinline|--split| option, for more details. \textbf{Warning:} when applied to a single target, it must be applied to \makeinline|$(AUX_DIR)/file/index.html:| instead of say \cmdinline|file.zip:|, otherwise zip and SCORM outputs will see different values. Default \texttt{\textit{section}}.
\item[\texttt{DVISVGM}] Command to call dvisvgm. Default \texttt{\textit{dvisvgm}}.
\item[\texttt{DVISVGMFLAGS}] Options to pass to dvisvgm. Default \texttt{\textit{--no-fonts}}.
\item[\texttt{LATEXMK}] Command to call Latexmk. Default \texttt{\textit{latexmk}}.
\item[\texttt{LATEXMKFLAGS}] Command options to pass to Latexmk. For instance, use \ltxinline|LATEXMKFLAGS=-pdflua| to use LuaTeX when compiling to PDF. Please ensure that Latexmk will produce a PDF rather than a DVI.
\item[\texttt{LATEXML}] Command to call \LaTeXML. Default \texttt{\textit{latexml}}.
\item[\texttt{LATEXMLFLAGS}] Options to pass to \LaTeXML.
\item[\texttt{LATEXMLPOST}] Command to call \texttt{latexmlpost}. Default \texttt{\textit{latexmlpost}}.
\item[\texttt{LATEXMLPOSTFLAGS}] Options to pass to \texttt{latexmlpost}. \textbf{Warning:} when applied to a single target, it must be applied to \ltxinline|$(AUX_DIR)/file/index.html:| instead of say \ltxinline|file.zip:|, just like for \texttt{SPLITAT}.
\item[\texttt{MUTOOL}] Command to call mutool. Default \texttt{\textit{mutool}}.
\item[\texttt{MUTOOLFLAGS}] Options to pass to mutool draw.
\item[\texttt{PDFTOSVG\_CONVERTER}] Select which converter to use for converting PDF images to SVG. Currently supported values are \texttt{dvisvgm} and \texttt{mutool}. If set to empty, LaTeXML will convert PDF to PNG using ImageMagick. Default \texttt{\textit{mutool}}.
\item[\texttt{PERL}] Command to call Perl. Default \texttt{\textit{perl}}.
\item[\texttt{TEXFOT}] Command to call texfot. Default \texttt{\textit{texfot}}.
\item[\texttt{TEXFOTFLAGS}] Options to pass to texfot.
\item[\texttt{ZIP}] Command to call zip. Default \texttt{\textit{zip}} (or \texttt{\textit{miktex-zip}} if \texttt{zip.exe} is not available on Windows).
\end{description}

\subsection{Makefile targets}\label{makefile-targets}
The following targets can be used as arguments when calling \texttt{make}, for instance \cmdinline{make zip}.

\begin{description}
\item[\texttt{all}] Compile all targets, based on the content of \makeinline|SOURCES|, \makeinline|FORMATS|, and \makeinline|TARGETS|. This is the default target.
\item[\texttt{check-for-update}] Check if there is a new release of BookML available.
\item[\texttt{clean}] Delete all compilation products, based on \makeinline|SOURCES|, \makeinline|FORMATS|, and \makeinline|TARGETS|.
\item[\texttt{detect}] Detect the versions of all the software required to run BookML and print them.
\item[\texttt{html}] Compile all \makeinline|SOURCES| to \HTML. The outputs will be in the \makeinline|$(AUX_DIR)/html| directory.
\item[\texttt{pdf}] Compile all \makeinline|SOURCES| to PDF. The outputs will be in the current directory, including the SyncTeX files.
\item[\texttt{scorm}] Compile all \makeinline|SOURCES| to SCORM. The outputs will be in the current directory.
\item[\texttt{update}] \textbf{Experimental:} update the \texttt{bookml/} directory. If BookML is being run from a Docker image, it will use the version bundled in the image, otherwise it will download the latest release. This operation is destructive, not well tested, and behaviour may change in the future; be prepared to download BookML manually if it breaks.
\item[\texttt{xml}] Compile all \makeinline|SOURCES| to XML. The outputs will be in the \makeinline|$(AUX_DIR)/xml| directory.
\item[\texttt{zip}] Compile all \makeinline|SOURCES| to zip. The outputs will be in the current directory.
\end{description}

\subsection{Custom \CSS}\label{custom-css}
The \CSS{} files in the folder \texttt{bmluser} are automatically included at the end of the \htmlinline|<head>| tag and will override the previous styles.

If the file name ends with \texttt{.style1,style2-jobname.css}, then that file will be used only when \ltxinline|style=style1| is passed, or when \ltxinline|style=style2| is passed and the main file being compiled is called \texttt{jobname.tex}. You can use \texttt{.\_all.css} to ensure that the file is included in every style.

Custom \CSS{} can also be added using \ltxinline|\bmlRawHTML{<style> ... </style>}| to the preamble.

For instance, the \href{https://vlmantova.github.io/bookml/index.plain.html}{plain} version of this manual has been compiled with \href{https://vlmantova.github.io/bookml/bmluser/latin-modern.plain.css}{\texttt{latin-modern.plain.css}} that sets the font to Latin Modern.

\subsection{Others}\label{others}
\begin{description}
  \item[\texttt{bml\_no\_invert}] Applying this \CSS{} class disable the color inversion of images in dark mode. This applies only to external images, such as EPS figures, or pictures converted using \ltxinline|bmlimage|. The class can be applied to the image itself or to a surrounding paragraph, section, environment, etc.\ using one of \ltxinline|\lxAddClass|, \ltxinline|\lxWithClass|, \ltxinline|\bmlPlusClass|.
\end{description}

\subsection{Options for the BookML GitHub action}\label{bookml-action}
Below is the list of arguments recognised by the BookML GitHub action. The list may be incomplete; see the \href{https://github.com/marketplace/actions/compile-with-bookml}{BookML action} page for details about the latest version.

\begin{description}
\item[\texttt{checkout}] Whether to checkout the repository calling this action. Default: \yamlinline|true| (boolean).
\item[\texttt{release}] Whether to create a release containing the outputs generated by BookML. Default: \yamlinline|true| (boolean).
\item[\texttt{upload-aux-directory}] Whether to upload the entire aux directory, which contains all outputs generated by BookML as well as logs and other intermediate files, into a GitHub artifact attached to the workflow run. Default: \yamlinline|true| (boolean).
\item[\texttt{scheme}] Select which TeX Live scheme to use among basic, small, medium, full. Default: \yamlinline|'full'| (string).
\item[\texttt{version}] Select which version of BookML to use. Note that this only affects which Docker image is used; if the \yamlinline|bookml/| folder is already present in the repository, that version of BookML will be used. Default: \yamlinline|'latest'| (string).
\item[\texttt{replace-bookml}] Whether to replace the \yamlinline|bookml/| folder with the one included in the Docker image. Default: \yamlinline|false| (boolean).
\item[\texttt{timeout-minutes}] The maximum number of minutes to run BookML before cancelling the build. Default: \yamlinline|6| (positive integer).
\end{description}

The action has the following outputs.

\begin{description}
  \item[\texttt{outputs}] File names of all outputs compiled by BookML.
  \item[\texttt{targets}] File names of all targets that BookML tried to compile.
  \item[\texttt{outcome}] Compiling outcome (one of \yamlinline|'success'|, \yamlinline|'failure'|, \yamlinline|'timeout'|, \yamlinline|'invalid'|, \yamlinline|'cancelled'|).
  \item[\texttt{aux-directory-url}] URL of GitHub artifact containing the aux directory (only if \yamlinline|upload-aux-directory| is true).
  \item[\texttt{release-url}] URL of GitHub release (only if \yamlinline|release| is true).
\end{description}

\end{document}
