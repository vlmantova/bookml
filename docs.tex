\documentclass[a4paper,british]{article}

\usepackage[T1]{fontenc}
\usepackage[utf8]{inputenc}

\usepackage[british]{babel}
\usepackage{bookml/bookml}
\usepackage{latexml}

% Google Search Console verification file
\iflatexml
\lxRequireResource[type=application/octet-stream]{googlee1709314ff1666e6.html}
\fi

% better fonts
\usepackage{lmodern}
\usepackage{microtype}

\bmlAltFormat{docs.pdf}{PDF (serif)}
\bmlAltFormat{docs-sans.pdf}{PDF (sans serif)}
\bmlAltFormat{docs-sans-large.pdf}{PDF (sans, large)}
\ifcsname bmlCrop\endcsname
\usepackage{crop}
\fi

\usepackage[pdfusetitle,colorlinks]{hyperref}
\usepackage{ifpdf}
\ifpdf % make build reproducible
\pdfinfoomitdate=1
\pdftrailerid{}
\pdfsuppressptexinfo=-1
\hypersetup{pdfinfo={Creator={},Producer={}}}
\fi

\usepackage[dvipsnames]{xcolor}
\usepackage{geometry}
\setlength{\parindent}{0pt}
\setlength{\parskip}{0.3em}

\iflatexml
\def\Xy{Xy}
\else
\usepackage[all]{xy}
\usepackage{tikz}
\usetikzlibrary{ducks}
\usepackage{animate}
\fi
\def\tikzname{Ti\emph{k}Z}

\bmlImageEnvironment{animateinline}

\usepackage{listings}
\colorlet{WCAGGreen}{OliveGreen!80!black} % a bit darker to get sufficient contrast
\lstset{basicstyle={\small\ttfamily},%
  keywordstyle={\color{blue}\bfseries},%
  keywordstyle=[2]{\color{MidnightBlue}\bfseries},%
  stringstyle={\color{red}},%
  commentstyle={\color{WCAGGreen}},%
  frame=none,
  xleftmargin=1em,xrightmargin=1em%
}

\lstdefinestyle{bookml}{language=[LaTeX]TeX,%
  texcsstyle=*{\color{blue}\bfseries},%
  texcsstyle=*[2]{\color{MidnightBlue}\bfseries},%
  moretexcs={xymatrix,ltxinline,href,RequirePackage},%
  moretexcs=[2]{LaTeXML,iflatexml,lxAddClass,lxWithClass,lxBeginTableHead,lxEndTableHead,lxFcn,lxID,lxPunct,lxContextTOC,lxNavbar,lxHeader,lxFooter,bmlImageEnvironment,bmlHTMLEnvironment,bmlRawHTML,bmlDescription,bmlPlusClass,bmlHTMLInlineEnvironment,bmlAltFormat}%
  }

\def\ltxinline{\lstinline[style=bookml]}
\def\htmlinline{\lstinline[language=html]}

\usepackage{amsmath}

\title{BookML\@: automated LaTeX to \href{https://bookdown.org/yihui/bookdown/html.html\#gitbook-style}{bookdown}-style HTML and SCORM, powered by \href{https://dlmf.nist.gov/LaTeXML/}{LaTeXML}}
\author{Vincenzo Mantova}

\date{3\textsuperscript{rd} October 2024}

\begin{document}

\maketitle

\begin{abstract}
  BookML is a small wrapper around \href{https://dlmf.nist.gov/LaTeXML/}{LaTeXML} for the production of accessible HTML content straight from LaTeX files, and for packaging it as SCORM\@. Created by and maintained for maths lecturers at the University of Leeds.
\end{abstract}

\begin{center}
  Sources available on \href{https://github.com/vlmantova/bookml/}{GitHub}.
  \bmlRawHTML{
    <br/>
    <a href="https://github.com/vlmantova/bookml/releases/latest"><img alt="Latest release" src="https://img.shields.io/github/v/release/vlmantova/bookml?logo=github\&amp;display\_name=release"/></a>
    <img alt="Download stats" src="https://img.shields.io/github/downloads/vlmantova/bookml/total?logo=github" class="noepub"/>
    <a href="https://github.com/vlmantova/bookml/stargazers"><img alt="Stargazers on GitHub" src="https://img.shields.io/github/stars/vlmantova/bookml?logo=github"/></a>
  }

  Formats: \href{https://vlmantova.github.io/bookml/}{GitBook} (html), \href{index.plain.html}{plain} with Latin Modern (html), \href{https://vlmantova.github.io/bookml/docs.pdf}{PDF}, \href{https://github.com/vlmantova/bookml/blob/docs/docs.tex}{\LaTeX{} source}.

  For a more beginner friendly guide see the \href{https://vlmantova.github.io/bookmlleeds/}{Leeds BookML guide}.
\end{center}

\tableofcontents

\section{Getting started}

\subsection{Prerequisites}\label{prerequisites}
\begin{itemize}
\item \href{https://dlmf.nist.gov/LaTeXML/get.html}{LaTeXML} (minimum 0.8.5, recommended 0.8.6 or later)
\item for any image handling: the Perl module \href{https://metacpan.org/pod/Image::Magick}{\texttt{Image::Magick}}
\item for handling EPS, PDF images: \href{https://www.ghostscript.com/}{Ghostscript}
\item for BookML images (for \tikzname{} and similar packages): \href{https://www.ghostscript.com/}{Ghostscript}, \href{https://ctan.org/pkg/latexmk}{latexmk}, \href{https://ctan.org/pkg/preview}{\texttt{preview.sty}}, \href{https://ctan.org/pkg/dvisvgm}{dvisvgm} (minimum 1.6, recommended 2.7 or later)
\item for automatic PDF, HTML, zip, SCORM packaging: \href{https://www.gnu.org/software/make/}{GNU make}, \href{https://ctan.org/pkg/latexmk}{latexmk}, \href{https://sourceforge.net/projects/infozip/}{zip}, optionally \href{https://ctan.org/pkg/texfot}{texfot}
\end{itemize}

\subsection{Installing BookML}
\begin{enumerate}
  \item Install the \hyperref[prerequisites]{prerequisites}.
  \item \textbf{Install/upgrade:} unpack the latest \href{https://github.com/vlmantova/bookml/releases}{BookML release} and put the \texttt{bookml} folder next to your \texttt{.tex} files.
  \item \textbf{First install only:} copy \texttt{bookml/GNUmakefile} next to your \texttt{.tex} files.
\end{enumerate}

Or you can unpack the \href{https://github.com/vlmantova/bookml/releases/latest/download/template.zip}{template} to start with a working minimal example.

The \href{https://vlmantova.github.io/bookmlleeds/}{Leeds BookML guide} has further examples and tips for lecturers and detailed installation instructions (some specific to the University of Leeds), including for instance how to compile exercises with and without solutions, or how to produce various alternative PDFs from the same file.

\subsection{Compilation}

Run \ltxinline|make| in the folder containing \texttt{GNUmakefile} and the \texttt{.tex} files. Each \texttt{.tex} file containing the string \ltxinline|\documentclass| will be compiled to PDF, \HTML{}, then a zip package and a SCORM package.

If compilation does not succeed, you may have to run \ltxinline|make| a second time. If it still fails, run \ltxinline|make clean-aux| or delete the \texttt{auxdir} folder to reset the state.

\section{Options}

The \ltxinline|bookml| package accepts a few options (for instance use \ltxinline|\usepackage[nomathjax]{bookml/bookml}| to avoid including MathJax). The options have no effect on the PDF output.

\begin{description}
  \item[\texttt{style=gitbook}] Use the GitBook style (the default behaviour). When using the GitBook style, you must call \ltxinline|latexmlc| (or \ltxinline|latexmlpost|) with the option \ltxinline|--navigationtoc=context|. Any PDF or EPUB file with the same name as the source will be detected and added to the download menu.
  \item[\texttt{style=plain}] Use the \LaTeXML{} style with a few slightly opinionated tweaks.
  \item[\texttt{style=none}] Use the \LaTeXML{} style with no tweaks (except for backported styles and some fixes).
  \item[\texttt{nomathjax}] Do not include MathJax in the output.
  \item[\texttt{mathjax=2}] Use MathJax version 2 instead of version 3.
  \item[\texttt{imagescale=X.XXX}] \textbf{(DEPRECATED)} Rescale the images generated via \LaTeX{} (\S~\ref{sec:external-image}) by the desired factor. The scaling factor is adjusted internally based on the options \ltxinline|8pt|, \ltxinline|9pt|, \dots, \ltxinline|12pt| being passed to the document class.
  \item[\texttt{nohtmlsyntax}] Disable the \HTML{} syntax that makes \ltxinline|\<| into the BookML command for introducing \HTML{} tags. Required if you already define \ltxinline|\<| for something else.
\end{description}

\section{Customisation}

\subsection{\CSS{} and fonts}
The \CSS{} files in the folder \lstinline|bmluser|, if one exists, are automatically included at the end of the \htmlinline|<head>| tag and will override the previous styles.

If the file name ends with \lstinline|.style1,style2-jobname.css|, then that file will be used only when \lstinline|style=style1| is passed, or when \lstinline|style=style2| is passed and the file is called \lstinline|jobname.tex|. You can use \lstinline|._all.css| to ensure that the file is included in every style.

The \href{https://vlmantova.github.io/bookml/index.plain.html}{plain} version of this manual has been compiled with \href{https://vlmantova.github.io/bookml/bmluser/latin-modern.plain.css}{\texttt{latin-modern.plain.css}} that sets the font to Latin Modern.

\subsection{\HTML{} output}

You can copy the file \texttt{bookml/LaTeXML-html5.xsl} one level up and modify it to customise the \HTML{} output. Knowledge of XSLT is required!

\subsection{Splitting}

By default, files are split into separate pages for each section. You can change this by running \lstinline|make SPLITAT=chapter| or \lstinline|make SPLITAT=|.

To make the change permanent, add the line \lstinline|SPLITAT=chapter| to \texttt{GNUmakefile}. You can also specify different values per file, for instance \lstinline|docs.zip: SPLITAT=| (or better yet, \lstinline|auxdir/html/docs/index.html: SPLITAT=|).

The \lstinline|latexml| command line can be further customize by setting the variables \lstinline|LATEXMLEXTRAFLAGS| and \lstinline|LATEXMLPOSTEXTRAFLAGS|. See the file \texttt{bookml/bookml.mk} for all the options and the default values.

\section{Commands}

\subsection{Conditional execution}
Call \ltxinline|\iflatexml ... \else ... \fi| to write code that is executed only by \LaTeXML{}, or only (pdf)\LaTeX{} respectively. \lstinline|bookml| will try to use the \lstinline|latexml| package, if available, or use its own embedded copy. See Figure~\ref{fig:iflatexml}.

\begin{figure}[hb]
  \begin{lstlisting}[style=bookml]
\caption{Example of
  \iflatexml\ltxinline|\xymatrix|\else\ltxinline|\\xymatrix|\fi{}
  from the \ltxinline|xypic| documentation.}
  \end{lstlisting}
  \caption{Example of \texttt{\textbackslash{}iflatexml}, used in Figure~\ref{fig:xymatrix} to work around a subtle difference between \LaTeXML{} and \LaTeX{}.}
  \label{fig:iflatexml}
\end{figure}

\subsection{Alternative text for images}
Call \ltxinline|\bmlDescription{textual description}| \emph{right after} an image to populate its \htmlinline|alt| attribute (or \ltxinline|aria-label| if appropriate). Inspect the \HTML{} source of Figure~\ref{fig:duck} or use a screen reader to check its text description.

\LaTeXML{} version 0.8.7 and later support the new \LaTeX{} syntax which (eventually) will also embed the alternative text in the PDF output: \ltxinline|\includegraphics[alt={textual description}]|.

\subsection{Disable MathJax for some equations}
Call \ltxinline|\bmlDisableMathJax{}| inside an equation to stop MathJax from processing the equation. Useful if you want to use MathJax, but you have equations that are better handled by the browser (you may need to include a suitable font, such as Latin Modern Math or STIX Two Math). When used inside an environment creating multiple equations, it applies only to the ones containing the command. See Figure~\ref{fig:disable-mathjax}.

\begin{figure}
  \begin{lstlisting}[style=bookml]
\begin{align*}
  \int_0^{+\infty} x^2 \mathrm{d}x \quad \text{rendered by \href{https://www.mathjax.org/}{MathJax}} \\
  \bmlDisableMathJax \int_0^{+\infty} x^2 \mathrm{d}x \quad \text{rendered by \emph{the browser}}
\end{align*}
  \end{lstlisting}
  \begin{align*}
    \int_0^{+\infty} x^2 \mathrm{d}x \quad \text{rendered by \href{https://www.mathjax.org/}{MathJax}} \\
    \bmlDisableMathJax \int_0^{+\infty} x^2 \mathrm{d}x \quad \text{rendered by \emph{the browser}}
  \end{align*}
  \caption{How to disable MathJax for a single equation.}
  \label{fig:disable-mathjax}
\end{figure}

\subsection{Add custom CSS classes}
Call \ltxinline|\bmlPlusClass{class}| \emph{right after} some piece of content to add a CSS class. If done within text, its effect may be unpredictable. Its main use is to call \ltxinline|\bmlPlusClass{bml_no_invert}| after an image to prevent the picture from getting inverted in night mode. Compare how Figure~\ref{fig:duck} (with \ltxinline|bml_no_invert|) and Figure~\ref{fig:xymatrix} (no additional classes) change in night mode to see the difference.

Note that the package \ltxinline|latexml| also offers \ltxinline|\lxAddClass| and \ltxinline|\lxWithClass| for the same effect but different behaviour regarding which element gets the class.

\subsection{Generate pictures with \LaTeX{}}
\label{sec:external-image}

\LaTeXML{} supports the \ltxinline|picture| environment as well as \tikzname{} pictures and some \Xy-matrices. In some situations, especially with \Xy-matrices, the output is can be mangled. Some common packages are not supported altogether (for instance \ltxinline|animate|).

BookML offers a simple automated way of generating SVG images using \LaTeX{}, bypassing \LaTeXML{} entirely. In your preamble, after \ltxinline|\usepackage{bookml/bookml}|, write
\begin{lstlisting}[style=bookml]
\bmlImageEnvironment{animateinline}

\usepackage{tikz}
\usepackage{tikzcd}
% optional, but speeds up compilation:
% do not load animate when running in LaTeXML
\iflatexml\else
\usepackage{animate}
\fi
\end{lstlisting}

All environments passed to \ltxinline|\bmlImageEnvironment|, in this case \ltxinline|animate|, will be compiled with \LaTeX{} using \lstinline[frame=none]|latexmk| and converted to SVG images via \lstinline[frame=none]|dvisvgm|. Figure~\ref{fig:duck} demonstrates this approach.

\begin{figure}
  \begin{lstlisting}[style=bookml]
\begin{animateinline}[
  alttext=none,loop,controls,nomouse,poster=20,autopause,
  begin={\begin{tikzpicture}
    \useasboundingbox (0,0) rectangle (5,3);},
    end={\end{tikzpicture}}]{30}
  \multiframe{160}{dShift=50mm+-0.5mm}
    {\duck[tophat,xshift=\dShift]}
\end{animateinline}
\bmlDescription{A stylised rubber duck, yellow
  and wearing a black top hat, enters from the right
  and slides until it exits from the left. The animation
  repeats every six seconds.}
\bmlPlusClass{bml_no_invert} % preserve colours in night mode
  \end{lstlisting}
  \begin{center}
    \begin{animateinline}[
      alttext=none,loop,controls,nomouse,poster=20,autopause,
      poster=20,autopause,
      begin={\begin{tikzpicture}%
        \useasboundingbox (0,0) rectangle (5,3);},
        end={\end{tikzpicture}}]{30}
      \multiframe{160}{dShift=50mm+-0.5mm}
        {\duck[tophat,xshift=\dShift]}
    \end{animateinline}
  \end{center}
  \bmlDescription{A stylised rubber duck, yellow
    and wearing a black top hat, enters from the right
    and slides until it exits from the left. The animation
    repeats this every six seconds.}
  \bmlPlusClass{bml_no_invert} % preserve colours in night mode
  \caption{A fancy duck. Click on the play button to start the animation (for the PDF, it requires a compatible software such as Acrobat Reader).}
  \label{fig:duck}
\end{figure}

If you only need this mechanism in a pinch, you can simply wrap the desired content between \ltxinline|\begin{bmlimage}| and \ltxinline|\end{bmlimage}| as exemplified in Figure~\ref{fig:xymatrix}.
\begin{figure}
  \begin{lstlisting}[style=bookml]
\[ \begin{bmlimage}
  \xymatrix{
    U \ar@/_/[ddr]_y \ar@/^/[drr]^x \ar@{.>}[dr]|-{(x,y)} \\
    & X \times_Z Y \ar[d]^q \ar[r]_p & X \ar[d]_f \\
    & Y \ar[r]^g & Z}
  \end{bmlimage} \]
  \end{lstlisting}
  \[ \begin{bmlimage}
    \xymatrix{
      U \ar@/_/[ddr]_y \ar@/^/[drr]^x \ar@{.>}[dr]|-{(x,y)} \\
      & X \times_Z Y \ar[d]^q \ar[r]_p & X \ar[d]_f \\
      & Y \ar[r]^g & Z}
    \end{bmlimage} \]
  \bmlDescription{Universal property of the pullback as defined in [omitted].}
  \caption{Example of \iflatexml\ltxinline|\xymatrix|\else\ltxinline|\\xymatrix|\fi{} from the \ltxinline|xypic| documentation.}
  \label{fig:xymatrix}
\end{figure}

\subsection{Alternative formats}
\ltxinline|\bmlAltFormat{docs.large.pdf}{PDF (large print)}| instructs BookML to compile \texttt{docs.large.tex} to PDF and include the result in the \HTML{} output. The file will appear in the `Downloads' menu of the GitBook style.

Please note that \texttt{docs.large.tex} must \emph{not} contain the string \ltxinline|\documentclass| or it will be compiled by itself into \HTML{}, zip, and SCORM.

See \texttt{template.tex}, \texttt{template-sans.tex}, \texttt{template-sans-large.tex} for an example.

\subsection{SCORM description}
BookML supports adding metadata to your SCORM packages via the \texttt{hyperref} package:
\begin{lstlisting}[style=bookml]
\usepackage{hyperref}
\hypersetup{pdfsubject={Description of this file}}
\end{lstlisting}
Note that the metadata is not specific to the SCORM package, but will also become part of the PDF. Use \ltxinline|\iflatexml...\fi| appropriately if you need different information. If the \texttt{pdfsubject} is not present, the abstract will be used as description.

\subsection{Direct \HTML{} input}

You can insert arbitrary \HTML{} code using \ltxinline|\bmlRawHTML{html code}|.

\textbf{Warning:} the \HTML{} code needs to be written in `XML syntax', so you have to close all the tags (for instance, write \htmlinline|<br/>| instead of \htmlinline|<br>|, close the \htmlinline|<p>| tags, and so on) and empty attributes \emph{must} be given the value \htmlinline|""| (see this \href{https://dev.w3.org/html5/html-author/#elements}{old W3C guide} for some indications). Moreover, you must remember to escape your \ltxinline|%&_^${}|, and replace \ltxinline|\| with \ltxinline|\textbackslash|.

\ltxinline|\bmlRawHTML| is robust, i.e.\ it does not change the category codes, so it can be used inside \ltxinline|\newcommand| to create custom macros. See for instance Figure~\ref{fig:youtube-ducks} for a generic YouTube embedding macro. Note that the video will not be visible in the PDF, so a link should always be provided (possibly PDF only, as in the example).

\begin{figure}
  \begin{lstlisting}[style=bookml]
\newcommand{\youtube}[2]{\bmlRawHTML{
  <div style="max-width: 1920px; width: 100\%">
    <div style="position: relative;
        padding-bottom: 56.25\%; height: 0; overflow: hidden;">
      <iframe width="1920" height="1080"
        src="https://www.youtube-nocookie.com/embed/#1"
        title="YouTube: #2" allowfullscreen=""
        style="border:none; position: absolute; top: 0; left: 0;
          right: 0; bottom: 0; height: 100\%; max-width: 100\%;"
        allow="accelerometer; autoplay; clipboard-write;
          encrypted-media; gyroscope; picture-in-picture"/>
    </div>
  </div>}
\iflatexml\else
\begin{center}
  Watch \href{https://www.youtube.com/watch?v=#1}{#2}.
\end{center}
\fi}
\youtube{mH0oCDa74tE}
  {Group theory, abstraction, and the 196,883-dimensional monster}
  \end{lstlisting}
  \newcommand{\youtube}[2]{\bmlRawHTML{
    <div style="max-width: 1920px; width: 100\%">
      <div style="position: relative;
          padding-bottom: 56.25\%; height: 0; overflow: hidden;">
        <iframe width="1920" height="1080"
          src="https://www.youtube-nocookie.com/embed/#1"
          title="YouTube: #2" allowfullscreen=""
          style="border:none; position: absolute; top: 0; left: 0;
            right: 0; bottom: 0; height: 100\%; max-width: 100\%;"
          allow="accelerometer; autoplay; clipboard-write;
            encrypted-media; gyroscope; picture-in-picture"/>
      </div>
    </div>}
  \iflatexml\else
  \begin{center}
    Watch \href{https://www.youtube.com/watch?v=#1}{#2}.
  \end{center}
  \fi}
  \youtube{mH0oCDa74tE}
    {Group theory, abstraction, and the 196,883-dimensional monster}
  \caption{Demonstration of \iflatexml\ltxinline|\bmlRawHTML|\else\ltxinline|\\bmlRawHTML|\fi{} within \iflatexml\ltxinline|\newcommand|\else\ltxinline|\\newcommand|\fi{} with a video from \href{https://www.youtube.com/3blue1brown}{3Blue1Brown}.}
  \label{fig:youtube-ducks}
\end{figure}

\subsection{Interspersing \LaTeX{} and \HTML{} (beta)}

You may write arbitrary \HTML{} using the syntax \ltxinline|\<html-tag>|. Just as for \ltxinline|\bmlRawHTML|, you need to use the \XML{} syntax.

If you need the command sequence \ltxinline|\<| for other purposes, load the BookML package with the option \texttt{nohtmlsyntax}. You may still intersperse the \HTML{} tags by declaring them first with \ltxinline|\bmlHTMLEnvironment{tag}|, and then using \ltxinline|\begin{h:tag} ... \end{h:tag}|. Attributes can be passed as optional arguments \ltxinline|\begin{h:tag}[attr1=val1,attr2=val2]|. You can specify multiple tags by separating them with commas, as in \ltxinline|\bmlHTMLEnvironment{tag1,tag2}|.

Use \ltxinline|\bmlHTMLInlineEnvironment{tag}| for tags that can only contain `phrasing' content, for instance they should not contain \htmlinline|<p>| paragraphs.

See figures~\ref{fig:details},~\ref{fig:details-nohtmlsyntax} for an example in both styles.

\begin{figure}
  \<details style="text-align: left; width: 100\%" open="">
    \<summary>
      \textbf{Code for \lstinline[language=html,frame=none]|<details>|.}
    \</summary>

    Completing the quine is left as an exercise for the reader.
    \begin{lstlisting}[style=bookml]
\<details style="text-align: left; width: 100\%" open="">
  \<summary>
    \textbf{Code for \lstinline[language=html,frame=none]|<details>|.}
  \</summary>

  Completing the quine is left as an exercise for the reader.
\</details>
      \end{lstlisting}
  \</details>
  \caption{Implementation of the \htmlinline|<details>| tag.}
  \label{fig:details}
\end{figure}

\begin{figure}
  \bmlHTMLEnvironment{details}
  \bmlHTMLInlineEnvironment{summary}
  \begin{h:details}[style={text-align: left; width: 100\%},open]
    \begin{h:summary}
      \textbf{Code for \lstinline[language=html,frame=none]|<details>|.}
    \end{h:summary}

    Completing the quine is left as an exercise for the reader.
    \begin{lstlisting}[style=bookml]
\bmlHTMLEnvironment{details}
\bmlHTMLInlineEnvironment{summary}
\begin{h:details}[style={text-align: left; width: 100\%},open]
  \begin{h:summary}
    \textbf{Code for \lstinline[language=html,frame=none]|<details>|.}
  \end{h:summary}

  Completing the quine is left as an exercise for the reader.
\end{h:details}
    \end{lstlisting}
  \end{h:details}
  \caption{Implementation of the \htmlinline|<details>| tag if using \texttt{nohtmlsyntax}.}
  \label{fig:details-nohtmlsyntax}
\end{figure}

\subsection{...and everything from \texttt{latexml.sty}}
By using \ltxinline|\usepackage{bookml/bookml}|, you will also import \texttt{latexml.sty}, making several \LaTeXML{}-related commands available, for instance \ltxinline|\lxBeginTableHead| for marking table headers. Please read the \href{https://github.com/brucemiller/LaTeXML/blob/81f2f004bdaa6860d673382837c4d4e3951a602e/lib/LaTeXML/texmf/latexml.sty}{source of \texttt{latexml.sty}} to learn what is included.

\end{document}
